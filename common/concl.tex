%% Согласно ГОСТ Р 7.0.11-2011:
%% 5.3.3 В заключении диссертации излагают итоги выполненного исследования, рекомендации, перспективы дальнейшей разработки темы.
%% 9.2.3 В заключении автореферата диссертации излагают итоги данного исследования, рекомендации и перспективы дальнейшей разработки темы.
\begin{enumerate}
\item Были измерены инвариантные спектры по поперечному импульсу, факторы ядернонй модификации и отношения выходов идентифицируемых заряженных адронов ($\pi^\pm$, $K^\pm$, $p$, $\bar{p}$).

\item На основе анализа инвариантных \pt и \mt спектров были получены значения температуры химического вымораживания $T$ и усредненные коллективные скорости $\left< u_T \right>$. Значения  $T$ не зависят от значений центральности и $N_{part}$.
Значения  $\left< u_T \right>$ плавно увеличиваются с увеличением значений $N_{part}$.
Это может свидетельствовать о том, что при столкновениях, характеризующихся большими значениями $N_{part}$ (центральные Cu+Au, Au+Au, U+U столкновения), коллективные эффекты выражены сильнее, чем при столкновениях с малыми значениями $N_{part}$ (\pal, \heau столкновения).

\item Измерены отношения \prot/$\pi$ и $K/\pi$.
В столкновениях с малыми значениями \Npart (коллизии \pal, \heau, \dau и периферийные столкновения больших систем) отношения \prot/$\pi$ близки к таковым, измеренным в \pp-столкновениях ($(p/\ пи)^{р+р}$). В столкновениях с большими \Npart (центральные столкновения Cu+Au, Au+Au и U+U) отношения \prot/$\pi$ достигают значения 0,8, что в 2,5 раза больше, чем $(p/\pi)^{ р+р}$. Поэтому при столкновениях тяжелых ионов (\cuau, \auau, \uu) отношения \prot/$\pi$ сильно зависят от центральности, а при малых \heau и \dau столкновениях виден только намек на центральность, которая незначительна в пределах неопределенности. В \pal-столкновениях зависимость от центральности отсутствует и значения \prot/$\pi$ согласуются с таковыми, измеренными в \pp-столкновениях, в пределах погрешностей.
Отношения $K/\pi$ во всех рассмотренных системах также не показывают зависимости от центральности.
Наблюдаемое поведение отношений $p/\pi$ и $K/\pi$ можно описать в рамках рекомбинационных моделей адронизации КГП.

\item Сравнение идентифицированных факторов ядерной модификации заряженных адронов показало, что в системах столкновений \dau, \heau, Cu+Au, Au+Au и U+U значения \rab согласуются при одном и том же числе нуклонов-участников \Npart.
Таким образом, можно сделать вывод, что идентифицированное рождение заряженных адронов не зависит от геометрии и размера системы и определяется величиной \Npart.
Наклон $R_{AB}(p_T)$ в коллизиях \pal более пологий, чем в коллизиях \heau и \dau при тех же значениях \Npart.
Значения \rab для протонов в \pal-столкновениях равны единице в диапазоне 1 ГэВ/с $< p_T <$3 ГэВ/с, а значения \rab для протонов, измеренные в \heau и \dau-столкновениях, в пределах погрешностей больше единицы. Это может свидетельствовать о том, что идентифицированные механизмы рождения заряженных адронов в \pal и в $d$/$^{3}$He+Au столкновениях имеют различия.
\end{enumerate}
