\pdfbookmark{Общая характеристика работы}{characteristic}             % Закладка pdf
\section*{Общая характеристика работы}

\newcommand{\actuality}{\pdfbookmark[1]{Актуальность}{actuality}\underline{\textbf{\actualityTXT}}}
\newcommand{\progress}{\pdfbookmark[1]{Разработанность темы}{progress}\underline{\textbf{\progressTXT}}}
\newcommand{\aim}{\pdfbookmark[1]{Цели}{aim}\underline{{\textbf\aimTXT}}}
\newcommand{\tasks}{\pdfbookmark[1]{Задачи}{tasks}\underline{\textbf{\tasksTXT}}}
\newcommand{\aimtasks}{\pdfbookmark[1]{Цели и задачи}{aimtasks}\aimtasksTXT}
\newcommand{\novelty}{\pdfbookmark[1]{Научная новизна}{novelty}\underline{\textbf{\noveltyTXT}}}
\newcommand{\influence}{\pdfbookmark[1]{Практическая значимость}{influence}\underline{\textbf{\influenceTXT}}}
\newcommand{\methods}{\pdfbookmark[1]{Методология и методы исследования}{methods}\underline{\textbf{\methodsTXT}}}
\newcommand{\defpositions}{\pdfbookmark[1]{Положения, выносимые на защиту}{defpositions}\underline{\textbf{\defpositionsTXT}}}
\newcommand{\reliability}{\pdfbookmark[1]{Достоверность}{reliability}\underline{\textbf{\reliabilityTXT}}}
\newcommand{\probation}{\pdfbookmark[1]{Апробация}{probation}\underline{\textbf{\probationTXT}}}
\newcommand{\contribution}{\pdfbookmark[1]{Личный вклад}{contribution}\underline{\textbf{\contributionTXT}}}
\newcommand{\publications}{\pdfbookmark[1]{Публикации}{publications}\underline{\textbf{\publicationsTXT}}}

{\actuality} Кварк-глюонная плазма (КГП) - это состояние вещества, которое существует при чрезвычайно высокой температуре ($>175$ МэВ) и плотности ($\sim 10$ Фм/$c$).  КГП состоит из асимптотически свободных сильно взаимодействующих кварков и глюонов, которые обычно находятся внутри атомных ядер или других адронов. Считается, что первые несколько микросекунд после Большого Взрыва вселенная находилась в состоянии КГП. Согласно квантовой хромодинамике, КГП может образовываться также и в столкновениях релятивистских тяжелых ионов. По мере расширения, КГП остывает и при достижении критической температуры происходит процесс адронизации - фазовый переход в нейтральную по цвету адронную материю.
\autocite{physica2020}

Одной из моделей адронизации является модель рекомбинации, согласно которой адроны образуются в результате объединения кварков, которые находятся рядом в фазовом пространстве. Поскольку связанные адронные состояния являются непертурбативными в рамках квантовой хромодинамики, описание процесса адронизации является чрезвычайно сложной задачей. В связи с этим особую важность приобретает анализ экспериментальных данных.

В 2002 году экспериментом PHENIX в столкновениях Au+Au при энергиях 130 ГэВ, а в последствии и при 200 ГэВ, было обнаружено аномально большое, по сравнению с протон-протонными столкновениями, отношение выходов (анти)протонов к выходам  \pipm-мезонов. В протон-протонных столкновениях в области поперечных импульсов \pt $\approx$3 ГэВ/с барионов рождается в 3 раза меньше, чем мезонов. Это связано с большими массами барионов и требованием ненулевого барионного числа для образования бариона. Однако экспериментом PHENIX было обнаружено, что в центральных Au+Au столкновениях барионы и мезоны рождаются примерно в равной пропорции. С увеличением центральности столкновений различие между результатами в протон-протонных взаимодействиях и Au+Au взаимодействиях уменьшается. Единственной моделью, способной объяснить данную аномалию оказалась модель рекомбинации – одна из моделей адронизации КГП.

Согласно Квантовой хромодинамике ожидалось, что в легких системах столкновений, таких как p+Al, \heau, d+Au, при энергии \sqsn=200 ГэВ условия, необходимые для образования КГП не достигаются. Однако в 2018 г экспериментом PHENIX были обнаружены эллиптические потоки заряженных частиц в p+Al, \heau, d+Au столкновениях, которые могут быть интерпретированы как признак образования КГП. 

Систематическое изучение рождения заряженных адронов в легких (p+Al, p/d/\heau) и тяжелых (Cu+Au, Au+Au и U+U) позволит изучить минимальные условия образования КГП
Таким образом, настоящая работа, посвященная исследованию особенностей рождения заряженных адрнов (\pipm, \Kpm, \prot, \aprot )в столкновениях p+Al, He+Au, Cu+Au   при \sqsn=200 ГэВ и в U+U столкновениях при \sqsn=193 ГэВ актуальна и является важной составляющей систематического изучения свойств КГП.

\begin{comment}
\ifsynopsis
Этот абзац появляется только в~автореферате.
Для формирования блоков, которые будут обрабатываться только в~автореферате,
заведена проверка условия \verb!\!\verb!ifsynopsis!.
Значение условия задаётся в~основном файле документа (\verb!synopsis.tex! для
автореферата).
\else
Этот абзац появляется только в~диссертации.
Через проверку условия \verb!\!\verb!ifsynopsis!, задаваемого в~основном файле
документа (\verb!dissertation.tex! для диссертации), можно сделать новую
команду, обеспечивающую появление цитаты в~диссертации, но~не~в~автореферате.
\fi
\end{comment}
% {\progress}
% Этот раздел должен быть отдельным структурным элементом по
% ГОСТ, но он, как правило, включается в описание актуальности
% темы. Нужен он отдельным структурынм элемементом или нет ---
% смотрите другие диссертации вашего совета, скорее всего не нужен.

{\aim} данной работы является изучение свойств  кварк-глюонной плазмы (температуры) путем измерения заряженных адронов в столкновениях \pal, \heau, \cuau\, при энергии \sqsn=200 ГэВ и в столкновениях \uu при энергии \sqsn=193 ГэВ.

Для~достижения поставленной цели необходимо было решить следующие {\tasks}:
\begin{enumerate}[beginpenalty=10000] % https://tex.stackexchange.com/a/476052/104425
 \item Измерить инвариантные спектры по поперечному импульсу для \pipm, \Kpm, \prot\,и \aprot \,в столкновениях  p+Al, $^{3}$He+Au, Cu+Au при энергии $\sqrt{s_{NN}}$=200 ГэВ и в столкновениях U+U при энергии $\sqrt{s_{NN}}$=193 ГэВ.
\item Измерить факторы ядерной модификации для \pipm, \Kpm, \prot и \aprot в столкновениях  p+Al, $^{3}$He+Au, Cu+Au при энергии $\sqrt{s_{NN}}$=200 ГэВ и в столкновениях U+U при энергии $\sqrt{s_{NN}}$=193 ГэВ.

\item Измерить отношения выходов \pim/\pip, \Km/\Kp, \prot/\aprot, \prot/\pip, \aprot/\pim, \Kp/\pip, \Km/\pim в столкновениях  p+Al, $^{3}$He+Au, Cu+Au при энергии $\sqrt{s_{NN}}$=200 ГэВ и в столкновениях U+U при энергии $\sqrt{s_{NN}}$=193 ГэВ.
\item Провести физическую интерпретацию результатов.
\end{enumerate}


{\novelty}
\begin{enumerate}[beginpenalty=10000] % https://tex.stackexchange.com/a/476052/104425
	\item Впервые измерены инвариантные спектры рождения по поперечному импульсу заряженных адронов (\pipm, \Kpm, \prot, \aprot) в столкновениях  p+Al, \heau, Cu+Au при энергии \sqsn=200 ГэВ и в столкновениях U+U при энергии \sqsn=193 ГэВ.
	\item Впервые получены факторы ядерной модификации для \pipm, \Kpm, \prot, \aprot\, в столкновениях  p+Al, \heau, Cu+Au при \sqsn=200 ГэВ и в столкновениях U+U при \sqsn=193 ГэВ
	\item 	Выпервые измерены отношения выходов \pim/\pip, \Km/\Kp, \prot/\aprot, \prot/\pip, \aprot/\pim, \Kp/\pip, \Km/\pim\,  в столкновениях  p+Al, \heau, Cu+Au при энергии \sqsn=200 ГэВ и в столкновениях U+U при энергии \sqsn=193 ГэВ.
\end{enumerate}

{\influence} \ldots

{\methods} \ldots

{\defpositions}
\begin{enumerate}[beginpenalty=10000] % https://tex.stackexchange.com/a/476052/104425
	\item 	Инвариантные спектры рождения по поперечному импульсу идентифицируемых заряженных адронов в столкновениях  p+Al,\heau, Cu+Au при энергии \sqsn =200 ГэВ и в столкновениях U+U при энергии \sqsn=193 ГэВ. \autocite{Gosele1999161,Lermontov}
	\item 	факторы ядерной модификации для \pipm, \Kpm, \prot, \aprot\, в столкновениях  p+Al, \heau, Cu+Au при \sqsn=200 ГэВ и в столкновениях U+U при \sqsn=193 ГэВ
		\begin{enumerate} 
			\item 	Значения факторов ядерной модификации, измеренные для одного типа частиц в различных системах столкновений при одинаковых значениях \Npart, совпадают в пределах погрешностей.
			\item В тяжелых системах столкновений  (Cu+Au, U+U) факторы ядерной модификации протонов повышены.
			\item В тяжелых системах столкновений  (Cu+Au, U+U) факторы ядерной модификации заряженных адронов проявляют зависимость от центральности. В периферических столкновениях значения \rab всех легких адронов 
			\item В столкновениях \heau  факторы ядерной модификации протонов повышены, в то время как факторы ядерной модификации мезонов близки к единице и совпадают в пределах погрешностей.
			\item 	В столкновениях p+Al значения \rab всех легких адронов (\pipm, \Kpm, \prot, \aprot, $\phi$, $K_s$, $K^{*0}$, $\eta$, $\omega$, $\pi^0$) близки к единице и совпадают в пределах погрешностей.
			\item 	Зависимость значений \rab от поперечного импульса (\pt), измеренная для для \Kpm, \pipm  в столкновениях p+Al имеет меньший угол наклона, чем в столкновениях \heau.
		\end{enumerate}
	
	\item Отношения выходов \pim/\pip, \Km/\Kp, \prot/\aprot, \prot/\pip, \aprot/\pim, \Kp/\Km в столкновениях  p+Al, $^{3}$He+Au, Cu+Au при энергии $\sqrt{s_{NN}}$=200 ГэВ и в столкновениях U+U при энергии $\sqrt{s_{NN}}$=193 ГэВ.
		\begin{enumerate}
			\item 	В тяжелых системах (Cu+Au, U+U) столкновений отношения \prot/\pip   и \aprot/\pim проявляют зависимость от центральности. 
			\item 	В легких системах столкновений (He+Au, p+Al) зависимость отношений \prot/\pip   и \aprot/\pim - от центральности не наблюдается
			\item 	Зависимость от центральности отношений \Kp/\pip и \Km/\pim не наблюдается во всех рассматриваемых системах столкновений.
		\end{enumerate}

\end{enumerate}


{\reliability} Измерения выходов заряженных адронов выполнялись согласно методике, принятой в коллаборации PHENIX. Результаты обсуждались на семинарах коллаборации PHENIX и международных конференциях «Ядро-2020» (Дубна, РФ), «Ядро-2021» (Санкт-Петербург, РФ) «PhysicA.SPb 2020» (Санкт-Петербург, РФ), «PhysicA.SPb 2021» (Санкт-Петербург, РФ), «PhysicA.SPb 2022» (Санкт-Петербург, РФ), «Lomonosov Conference» (Москва, РФ), «ICPPA-2020», (Москва, РФ). Результаты опубликованы в рецензируемых журналах баз ВАК и SCOPUS/WebOfScience.


{\probation}
Основные результаты работы докладывались~на:
Основные результаты работы докладывались~на:
международных конференциях «Ядро-2020» (Дубна, РФ), «Ядро-2021» (Санкт-Петербург, РФ) «PhysicA.SPb 2020» (Санкт-Петербург, РФ), «PhysicA.SPb 2021» (Санкт-Петербург, РФ), «PhysicA.SPb 2022» (Санкт-Петербург, РФ), «Lomonosov Conference» (Москва, РФ), «ICPPA-2020», (Москва, РФ). Результаты опубликованы в рецензируемых журналах баз ВАК и SCOPUS/WebOfScience.


{\contribution} Автор внес определяющий вклад в работу по отбору и анализу данных, получению физического результата и оценке систематических неопределенностей измерений.

\ifnumequal{\value{bibliosel}}{0}
{%%% Встроенная реализация с загрузкой файла через движок bibtex8. (При желании, внутри можно использовать обычные ссылки, наподобие ``).
    {\publications} Основные результаты по теме диссертации изложены
    в~XX~печатных изданиях,
    X из которых изданы в журналах, рекомендованных ВАК,
    X "--- в тезисах докладов.
}%
{%%% Реализация пакетом biblatex через движок biber
    \begin{refsection}[bl-author, bl-registered]
        % Это refsection=1.
        % Процитированные здесь работы:
        %  * подсчитываются, для автоматического составления фразы "Основные результаты ..."
        %  * попадают в авторскую библиографию, при usefootcite==0 и стиле `\insertbiblioauthor` или `\insertbiblioauthorgrouped`
        %  * нумеруются там в зависимости от порядка команд `\printbibliography` в этом разделе.
        %  * при использовании `\insertbiblioauthorgrouped`, порядок команд `\printbibliography` в нём должен быть тем же (см. biblio/biblatex.tex)
        %
        % Невидимый библиографический список для подсчёта количества публикаций:
        \printbibliography[heading=nobibheading, section=1, env=countauthor,          keyword=biblioauthorvak]%
        \printbibliography[heading=nobibheading, section=1, env=countauthorwos,          keyword=biblioauthorwos]%
        \printbibliography[heading=nobibheading, section=1, env=countauthorscopus,       keyword=biblioauthorscopus]%
        \printbibliography[heading=nobibheading, section=1, env=countauthorconf,         keyword=biblioauthorconf]%
        \printbibliography[heading=nobibheading, section=1, env=countauthorother,        keyword=biblioauthorother]%
        \printbibliography[heading=nobibheading, section=1, env=countregistered,         keyword=biblioregistered]%
        \printbibliography[heading=nobibheading, section=1, env=countauthorpatent,       keyword=biblioauthorpatent]%
        \printbibliography[heading=nobibheading, section=1, env=countauthorprogram,      keyword=biblioauthorprogram]%
        \printbibliography[heading=nobibheading, section=1, env=countauthor,             keyword=biblioauthor]%
        \printbibliography[heading=nobibheading, section=1, env=countauthorvakscopuswos, filter=vakscopuswos]%
        \printbibliography[heading=nobibheading, section=1, env=countauthorscopuswos,    filter=scopuswos]%
        %
        %\nocite{confbib1}
        %
        {\publications} Основные результаты по теме диссертации изложены в~\arabic{citeauthor}~печатных изданиях,
        \arabic{citeauthorvak} из которых изданы в журналах, рекомендованных ВАК\sloppy%
        \ifnum \value{citeauthorscopuswos}>0%
            , \arabic{citeauthorscopuswos} "--- в~периодических научных журналах, индексируемых Web of~Science и Scopus\sloppy%
        \fi%
        \ifnum \value{citeauthorconf}>0%
            , \arabic{citeauthorconf} "--- в~тезисах докладов.
        \else%
            .
        \fi%
        \ifnum \value{citeregistered}=1%
            \ifnum \value{citeauthorpatent}=1%
                Зарегистрирован \arabic{citeauthorpatent} патент.
            \fi%
            \ifnum \value{citeauthorprogram}=1%
                Зарегистрирована \arabic{citeauthorprogram} программа для ЭВМ.
            \fi%
        \fi%
        \ifnum \value{citeregistered}>1%
            Зарегистрированы\ %
            \ifnum \value{citeauthorpatent}>0%
            \formbytotal{citeauthorpatent}{патент}{}{а}{}\sloppy%
            \ifnum \value{citeauthorprogram}=0 . \else \ и~\fi%
            \fi%
            \ifnum \value{citeauthorprogram}>0%
            \formbytotal{citeauthorprogram}{программ}{а}{ы}{} для ЭВМ.
            \fi%
        \fi%
        % К публикациям, в которых излагаются основные научные результаты диссертации на соискание учёной
        % степени, в рецензируемых изданиях приравниваются патенты на изобретения, патенты (свидетельства) на
        % полезную модель, патенты на промышленный образец, патенты на селекционные достижения, свидетельства
        % на программу для электронных вычислительных машин, базу данных, топологию интегральных микросхем,
        % зарегистрированные в установленном порядке.(в ред. Постановления Правительства РФ от 21.04.2016 N 335)
    \end{refsection}%
    \begin{refsection}[bl-author, bl-registered]
        % Это refsection=2.
        % Процитированные здесь работы:
        %  * попадают в авторскую библиографию, при usefootcite==0 и стиле `\insertbiblioauthorimportant`.
        %  * ни на что не влияют в противном случае
        \nocite{physica2020}%vak
    \end{refsection}%
        %
        % Всё, что вне этих двух refsection, это refsection=0,
        %  * для диссертации - это нормальные ссылки, попадающие в обычную библиографию
        %  * для автореферата:
        %     * при usefootcite==0, ссылка корректно сработает только для источника из `external.bib`. Для своих работ --- напечатает "[0]" (и даже Warning не вылезет).
        %     * при usefootcite==1, ссылка сработает нормально. В авторской библиографии будут только процитированные в refsection=0 работы.
}

\begin{comment}
При использовании пакета \verb!biblatex! будут подсчитаны все работы, добавленные
в файл \verb!biblio/author.bib!. Для правильного подсчёта работ в~различных
системах цитирования требуется использовать поля:
\begin{itemize}
        \item \texttt{authorvak} если публикация индексирована ВАК,
        \item \texttt{authorscopus} если публикация индексирована Scopus,
        \item \texttt{authorwos} если публикация индексирована Web of Science,
        \item \texttt{authorconf} для докладов конференций,
        \item \texttt{authorpatent} для патентов,
        \item \texttt{authorprogram} для зарегистрированных программ для ЭВМ,
        \item \texttt{authorother} для других публикаций.
\end{itemize}
Для подсчёта используются счётчики:
\begin{itemize}
        \item \texttt{citeauthorvak} для работ, индексируемых ВАК,
        \item \texttt{citeauthorscopus} для работ, индексируемых Scopus,
        \item \texttt{citeauthorwos} для работ, индексируемых Web of Science,
        \item \texttt{citeauthorvakscopuswos} для работ, индексируемых одной из трёх баз,
        \item \texttt{citeauthorscopuswos} для работ, индексируемых Scopus или Web of~Science,
        \item \texttt{citeauthorconf} для докладов на конференциях,
        \item \texttt{citeauthorother} для остальных работ,
        \item \texttt{citeauthorpatent} для патентов,
        \item \texttt{citeauthorprogram} для зарегистрированных программ для ЭВМ,
        \item \texttt{citeauthor} для суммарного количества работ.
\end{itemize}
% Счётчик \texttt{citeexternal} используется для подсчёта процитированных публикаций;
% \texttt{citeregistered} "--- для подсчёта суммарного количества патентов и программ для ЭВМ.

Для добавления в список публикаций автора работ, которые не были процитированы в
автореферате, требуется их~перечислить с использованием команды \verb!\nocite! в
\verb!Synopsis/content.tex!.
\end{comment} % Характеристика работы по структуре во введении и в автореферате не отличается (ГОСТ Р 7.0.11, пункты 5.3.1 и 9.2.1), потому её загружаем из одного и того же внешнего файла, предварительно задав форму выделения некоторым параметрам

%Диссертационная работа была выполнена при поддержке грантов \dots

%\underline{\textbf{Объем и структура работы.}} Диссертация состоит из~введения,
%четырех глав, заключения и~приложения. Полный объем диссертации
%\textbf{ХХХ}~страниц текста с~\textbf{ХХ}~рисунками и~5~таблицами. Список
%литературы содержит \textbf{ХХX}~наименование.

\pdfbookmark{Содержание работы}{description}                          % Закладка pdf
\section*{Содержание работы}

Во \underline{\textbf{введении}} приводится обзор научной литературы по изучаемой проблеме, обосновывается актуальность исследований, проводимых в рамках данной диссертационной работы, ставятся задачи, формулируются цель, научная новизна и практическая значимость представляемой работы.
%Во введении обосновывается актуальность исследований, проводи­ мых в рамках данной диссертационной работы, приводится обзор научной литературы по изучаемой проблеме, формулируется цель, ставятся зада­чи работы, сформулированы научная новизна и практическая значимость представляемой работы.

В \underline{\textbf{первой главе}} основной части кратко изложены основные положения теории столкновения релятивистских ионов.  Приведены описание геометрии и принципиальная схема эволюции ядро-ядерных взаимодействий. Рассматрены особенности рождения легких адронов как в столкновениях тяжелых систем (таких как Cu+Au и U+U) так и в столкновениях легких систем (таких как $p$+Al, $^{3}$He+Au). Особое внимение уделено описанию проблем деконфайнмента КХД материи и фазовому переходу между КГП и адронным газом. Приведены основные наблюдаемые признаки рождения КГП в столкновениях тяжелых ядер. Рассмотрены основные результаты, полученные в экспериментах на RHIC и LHC, в том числе результаты измерения факторов ядерной модификации заряженных адронов в столкновениях ультрарелятивистских тяжелых ядер.

\underline{\textbf{Вторая глава}} посвящена описанию коллайдера релятивистских тяжелых ионов RHIC и основных детекторных подсистем спектрометра PHENIX. Приведено описание триггеров реального времени, используемых для получения выборок данных. Рассмотрены конструкционные особенности времяпролетной и дрейфовой камер, используемых для регистрации и идентификации заряженных адронов.

В \underline{\textbf{третьей главе}} основной части описаны критерии отбора данных и методика, с помощью которой были получены результаты.

В данной работе были использованы данные, полученные экспериментом PHENIX в столкновениях $p$+$p$ (2005 год набора данных), Cu+Au (2012 год набора данных), $^{3}$He+Au (2014 год набора данных) и $p$+Al (2015 год набора данных) при энергии $\sqrt{s_{NN}}=200$ГэВ/$c$, а так же в столкновениях U+U (2012 год набора данных) при энергии $\sqrt{s_{NN}}=193$ГэВ/$c$. Измерения проводились в центральном диапазоне псевдобастрот $|\eta|<0.35$.
Были использованы критерии отбора данных, общепринятные в эксперименте PHENIX, такие как выбор событий с минимальным отбором, ограничения z-координаты вершины столкновения и выбор бита качества трека частиц. 

Идетификация заряженных адронов проводилась с помощью метода времени пролета \autocite{nucleus2020}, \cite{PPG026, ppg146}. Согласно данному методу  квадрат массы частиц может быть определен в соответствии с выражением 
$$m^2 = \frac{p^2}{c^2} \left(  \frac{t^2 c^2}{L^2} - 1\right)$$
где $p$ -- импульс частицы, измеренный с помощью дрейфовой камеры, $L$ -- длина, пройденная частицей в сцинтилляторной планке TOF, $t$ -- время регистрации частицы в TOF,  $c$ -- скорость света.
%Про 2 сигма

Оценки эффективности регистрации заряженных адронов проведена с помощью метода Монте-Карло, реализованного в программном пакете PISA \cite{PISA}. Полученная эффективность регистрации адронов в дальнейшем использовалась для расчета инвариантных спектров. 

Приведена классификация и оценка систематических неопределенностей. %Получено, что систематические неопределенности не превышают ...

\begin{comment}
Формулы в строку без номера добавляются так:
\[
    \lambda_{T_s} = K_x\frac{d{x}}{d{T_s}}, \qquad
    \lambda_{q_s} = K_x\frac{d{x}}{d{q_s}},
\]
\end{comment}

В \underline{\textbf{четвертой главе}} приведены результаты измерения инвариантных спектров по поперечному импульсу и факторы ядерной модифика цииидентифицированных заряженных адронов (\pipm, \Kpm, \prot, \aprot),а также величины отношений \pim/\pip, \Km/\Kp, \prot/\aprot, \prot/\pip, \aprot/\pim, \Kp/\pip, \Km/\pim в столкновениях  \pal, \heau, Cu+Au при энергии \sqsn=200 ГэВ и в столкновениях U+U при энергии \sqsn=193 ГэВ.

Представлены сравнения факторов ядерной модификации заряженных адронов в тяжелых (Cu+Au, Au+Au, U+U) и легких (\pal, \dau, \heau) системах столкновений.
На основании сравнения результатов, полученных в столкновениях с разной геометрии, сделан вывод о том, что процесс рождения идентифицируемых заряженных адронов не зависит от формы перекрытия сталкивающихся ядер, а определяется величиной \Npart.

В столкновениях, характеризующихся большими значениями \Npart (\Npart$>10$) был обнаружен эффект увеличенного выхода протонов и антипротонов, считающийся одинм из признаков образования КГП.

\underline{\textbf{Пятая глава}} основной части посвящена обсуждению полученных результатов. 

Путем построения и анализа инвариантных спектров рождения идентифицированных заряженных адронов по поперечной массе ($m_T = \sqrt{p_T^2 +m_h^2}$) особенности рождения \pipm, \Kpm, \prot, \aprot \ были интерпретированны в рамках модели радиально расширяющейся термализованной системы. Согласно данной модели, кинетическая энергия частиц складывается из энергии их теплового движения $T_0$ и энергии, приобретаемой частицами за счет радиального расширения системы ($m_h\left< u_t\right>$). Параметры $T_0$ и \ut \  интерпретируются как температура <<вымораживания>> системы и средняя скорость ее расширения соответственно. Параметры $T_{0}$ и \ut \ были вычисленны в \pal, \heau, Cu+Au и U+U столкновениях при различных значениях \Npart. 
Полученные зависимости $T_{0}(N_{part})$ и \ut(\Npart) приведены на рис. \ref{fig:synops_T0ut}. Величина $T_{0}\approx170$ МэВ и является постоянной относительно значений \Npart, в то время как величины \ut \ увеличиваются с увеличением значений \Npart.

\begin{figure}[ht]
	\centerfloat{
		\hfill
		\subcaptionbox{ }{%
			\includegraphics[width=0.45\linewidth]{Results/T0_Npart.png}}
		\hfill
		\subcaptionbox{ }{%
			\includegraphics[width=0.45\linewidth]{Results/ut_Npart.png}}
		\hfill
	}
	\caption{Зависимость а) температуры вымораживания $T_0$ и  б) скорости поперечного потока $u_T$ от количество нуклонов участников \Npart.}\label{fig:synops_T0ut}
\end{figure}
Для более глубокого анализа особенностей рождения идентифицированных заряженных адронов было проведено сравнение измеренных значений отношений  \pim/\pip, \Km/\Kp, \prot/\aprot, \prot/\pip, \aprot/\pim, \Kp/\pip, \Km/\pim с предсказаниями модели рекомбинации, реализованной в пакете AMPT в версии с плавлением струн (AMPTsm)\cite{AMPT}, а также модели фрагментации, реализованной в пакете PYTHIA8.3/ANGANTYR \cite{pythia}.
%Для более глубокого анализа полученных экспериментальных данных, было осуществлено моделирование \pal, \heau, Cu+Au столкновений при \sqsn \= 200 ГэВ и U+U столкновений при \sqsn \= 193 ГэВ с помощью программных пакетов PYTHA8.3/ANGANTYR \cite{pythia} и AMPT \cite{AMPT}. 
Результаты представлены на рис. \ref{img:synops_Ratio_LargeP2PI_sym}-\ref{img:synops_Ratio_SmallK2PI_sym}.
На основе сравнения эеспериментальных результатов и предсказаний моделей мделаны следующие выводы:

\begin{enumerate}
	\item Предсказания значений $p/\pi$ моделью, использующей фрагменационный механизм адронизации (PYTHIA8.3/ANGANTYR), совпадают в пределах погрешностей с экспериментальными результатами, полученными в $p$+$p$ столкновениях. 
	\item Рекомбинационная модель позволяет качественно (но не количественно) описать рост значений $p/\pi$, наблюдаемый в эксперименте в столкновениях, характеризующихся значениями количества нуклонов-участников \Npart $\gtrsim$10, а также зависимость этих значений от центральности. К столкновениям, характеризующимся значениями \Npart $\gtrsim$10 относятся центральные столкновения \heau, Cu+Au и U+U.
	\item Величины $p/\pi$, полученные с помощью рекомбинационной модели, реализованной в пакете AMPTsm, численно не совпадают с экспериментально измеренными. 
	\item В столкновениях, характеризующихся значениями количества нуклонов-участников \Npart $\lesssim$10, значения $p/\pi$, полученные с помощью рекомбинационной модели, совпадают с предсказаниями модели фрагментации в пределах погрешностей. К столкновениям с \Npart $\lesssim$10 относятся столкновения $p$+$p$, \pal, а также перефирические столкновения \heau, Cu+Au и U+U.
	\item Предсказания величин $K/\pi$ рекомбинационной моделью хорошо согласуются с экспериментальными результатами, в то время как модель фрагментации недооценивает величины $K/\pi$.
\end{enumerate}

%Согласно екомбинационной модели, $p_T$ произведенного адрона представляет собой сумму $p_T$ составляющих его кварков. Поэтому барионно-инвариантные спектры $p_T$ сдвинуты относительно мезонных спектров $p_T$ в сторону больших $p_T$, что приводит к увеличению производства барионов по сравнению с производством мезонов в промежуточной области $p_T$.
%Общее число барионов и мезонов, образующихся в процессах рекомбинации, пропорционально числу партонов внутри QGP.
%Следовательно, рост значений $p/\pi$ с увеличением центральности Cu+Au и U+U может свидетельствовать об усилении влияния рекомбинационных процессов и увеличении числа партонов в столкновении QGP. Согласие между предсказаниями модели рекомбинации и модели фрагментации в столкновениях с \Npart $\lesssim$10 может указывать на то, что объем образующегося QGP недостаточен для наблюдаемого увеличения выхода барионов.

\begin{comment}
\begin{figure}[] 
	\centerfloat
	\includegraphics [width=0.8\linewidth]{Simulation/RAA_AMPT_Pythia.png}
	\caption{Инвариантные спектры по поперечному массе, измеренные для отрицательно заряженных адронов в различных центральностях p+Al, \heau, Cu+Au и U+U столкновениях.} 
	\label{img:synops_RAA_sym}
\end{figure}


\begin{figure}[] 
	\centerfloat
	\includegraphics [width=0.6\linewidth]{Simulation/Ratio_same_AMPT_Pythia.png}
	\caption{Инвариантные спектры по поперечному массе, измеренные для отрицательно заряженных адронов в различных центральностях p+Al, \heau, Cu+Au и U+U столкновениях.} 
	\label{img:synops_Ratio_same_sym}
\end{figure}
\end{comment}

\begin{figure}[] 
	\centerfloat
	\includegraphics [width=0.7\linewidth]{Simulation/Ratios_AMPT_large_p2pi.png}
	\caption{Инвариантные спектры по поперечному массе, измеренные для отрицательно заряженных адронов в различных центральностях p+Al, \heau, Cu+Au и U+U столкновениях.} 
	\label{img:synops_Ratio_LargeP2PI_sym}
\end{figure}

\begin{figure}[] 
	\centerfloat
	\includegraphics [width=1\linewidth]{Simulation/Ratios_AMPT_small_p2pi.png}
	\caption{Инвариантные спектры по поперечному массе, измеренные для отрицательно заряженных адронов в различных центральностях p+Al, \heau, Cu+Au и U+U столкновениях.} 
	\label{img:synops_Ratio_SmallP2PI_sym}
\end{figure}

\begin{figure}[] 
	\centerfloat
	\includegraphics [width=0.7\linewidth]{Simulation/Ratios_AMPT_large_K2pi.png}
	\caption{Инвариантные спектры по поперечному массе, измеренные для отрицательно заряженных адронов в различных центральностях p+Al, \heau, Cu+Au и U+U столкновениях.} 
	\label{img:synops_Ratio_LargeK2PI_sym}
\end{figure}

\begin{figure}[] 
	\centerfloat
	\includegraphics [width=0.7\linewidth]{Simulation/Ratios_AMPT_small_K2pi.png}
	\caption{Инвариантные спектры по поперечному массе, измеренные для отрицательно заряженных адронов в различных центральностях p+Al, \heau, Cu+Au и U+U столкновениях.} 
	\label{img:synops_Ratio_SmallK2PI_sym}
\end{figure}


\FloatBarrier
\pdfbookmark{Заключение}{conclusion}                                  % Закладка pdf
В \underline{\textbf{заключении}} приведены основные результаты работы, которые заключаются в следующем:
%% Согласно ГОСТ Р 7.0.11-2011:
%% 5.3.3 В заключении диссертации излагают итоги выполненного исследования, рекомендации, перспективы дальнейшей разработки темы.
%% 9.2.3 В заключении автореферата диссертации излагают итоги данного исследования, рекомендации и перспективы дальнейшей разработки темы.
Представлены измерения инвариантных спектров по поперечному импульсу, факторов ядерной модификации, измеренные для идентифицируемых заряженных адронов ($\pi^\pm$, $K^\pm$, $p$, $\bar{p}$), а также отношений выходов адронов -- \pim/\pip, \Km/\Kp, \prot/\aprot, \prot/\pip, \aprot/\pim, \Kp/\pip, \Km/\pim.

На основе анализа инвариантных \pt \ и \mt \ спектров были получены значения температуры химического вымораживания $T_{0}$ и средней скорости коллективного потока частиц $\left< u_T \right>$ как функций от количества нуклонов-участников \Npart.
Величина $T_{0}\approx170$ МэВ и является постоянной относительно значений \Npart, в то время как величина \ut \ увеличивается с увеличением значений \Npart. Данный результат может свидетельствовать о том, что при столкновениях, характеризующихся большими значениями $\left<N_{part}\right>$ (центральные Cu+Au, Au+Au, U+U столкновения), коллективные эффекты выражены сильнее, чем при столкновениях с малыми значениями $N_{part}$ (\pal, \heau \ столкновения).

Сравнение идентифицированных факторов ядерной модификации заряженных адронов показало, значения \rab, измеренные в системах с разной геометрией (\dau, \heau, Cu+Au, Au+Au и U+U) совпадают при одинаковых значениях \Npart.
Сделан вывод, что рождение идентифицированных заряженных адронов не зависит от геометрии и размера системы столкновения и определяется величиной лишь размером области перекрытия ядер, характеризующегося величиной \Npart.

В центральных столкновениях \heau, Cu+Au, U+U был обнаружен эффект увеличенного выхода протонов и антипротонов, что было объяснено доминированием вклада процессов рекомбинации в образовние иднентифицируемых заряженных адронов в диапазоне малых и промежуточных поперечных импульсов ($p_{T}<4$ ГэВ/$c$). 
В \pal \ столкновениях, а также в периферических столкновениях \heau, Cu+Au, U+U эффект увеличенного выхода протонов и антипротонов не наблюдалтся, что было интерпретировано доминированием вклада процессов фрагментации в образовние иднентифицируемых заряженных адронов в диапазоне промежуточных поперечных импульсов (2 ГэВ/$c$ $<p_{T}<4$ ГэВ/$c$).

Полученные значения инвариантных спектров заряженных адронов могут быть использованы для уточнения параметров теоретических моделей, реализованных в пакетах прикладных программ, таких как  AMPT, HIJING, PHSD и др. В частности, для уточнения радиуса рекомбинации в рекомбинационных моделях, реализованных в таких программных пакетах как AMPT, PHSD.

\pdfbookmark{Литература}{bibliography}                                % Закладка pdf

\ifdefmacro{\microtypesetup}{\microtypesetup{protrusion=false}}{} % не рекомендуется применять пакет микротипографики к автоматически генерируемому списку литературы
\urlstyle{rm}                               % ссылки URL обычным шрифтом
\ifnumequal{\value{bibliosel}}{0}{% Встроенная реализация с загрузкой файла через движок bibtex8
    \renewcommand{\bibname}{\large \bibtitleauthor}
    \nocite{*}
    \insertbiblioauthor           % Подключаем Bib-базы
    %\insertbiblioexternal   % !!! bibtex не умеет работать с несколькими библиографиями !!!
}{% Реализация пакетом biblatex через движок biber
    % Цитирования.
    %  * Порядок перечисления определяет порядок в библиографии (только внутри подраздела, если \insertbiblioauthorgrouped).
    %  * Если не соблюдать порядок "как для \printbibliography", нумерация в \insertbiblioauthor будет кривой.
    %  * Если цитировать каждый источник отдельной командой --- найти некоторые ошибки будет проще.
    %
    %% authorvak
   \nocite{physica2020}%vak
   \nocite{physica2021}%vak
   \nocite{icppa2020}%vak
   \nocite{lomcon2021}%vak
   \nocite{nucleus2020}%vak    

    \ifnumgreater{\value{usefootcite}}{0}{
        \begin{refcontext}[labelprefix={}]
            \ifnum \value{bibgrouped}>0
                \insertbiblioauthorgrouped    % Вывод всех работ автора, сгруппированных по источникам
            \else
                \insertbiblioauthor      % Вывод всех работ автора
            \fi
        \end{refcontext}
    }{
        \ifnum \totvalue{citeexternal}>0
            \begin{refcontext}[labelprefix=A]
                \ifnum \value{bibgrouped}>0
                    \insertbiblioauthorgrouped    % Вывод всех работ автора, сгруппированных по источникам
                \else
                    \insertbiblioauthor      % Вывод всех работ автора
                \fi
            \end{refcontext}
        \else
            \ifnum \value{bibgrouped}>0
                \insertbiblioauthorgrouped    % Вывод всех работ автора, сгруппированных по источникам
            \else
                \insertbiblioauthor      % Вывод всех работ автора
            \fi
        \fi
        %  \insertbiblioauthorimportant  % Вывод наиболее значимых работ автора (определяется в файле characteristic во второй section)
        \begin{refcontext}[labelprefix={}]
            \insertbiblioexternal            % Вывод списка литературы, на которую ссылались в тексте автореферата
        \end{refcontext}
        % Невидимый библиографический список для подсчёта количества внешних публикаций
        % Используется, чтобы убрать приставку "А" у работ автора, если в автореферате нет
        % цитирований внешних источников.
        \printbibliography[heading=nobibheading, section=0, env=countexternal, keyword=biblioexternal, resetnumbers=true]%
    }
}
\ifdefmacro{\microtypesetup}{\microtypesetup{protrusion=true}}{}
\urlstyle{tt}                               % возвращаем установки шрифта ссылок URL
