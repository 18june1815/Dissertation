%% Согласно ГОСТ Р 7.0.11-2011:
%% 5.3.3 В заключении диссертации излагают итоги выполненного исследования, рекомендации, перспективы дальнейшей разработки темы.
%% 9.2.3 В заключении автореферата диссертации излагают итоги данного исследования, рекомендации и перспективы дальнейшей разработки темы.
Представлены измерения инвариантных спектров по поперечному импульсу, факторов ядерной модификации, измеренные для идентифицируемых заряженных адронов ($\pi^\pm$, $K^\pm$, $p$, $\bar{p}$), а также отношений выходов адронов -- \pim/\pip, \Km/\Kp, \prot/\aprot, \prot/\pip, \aprot/\pim, \Kp/\pip, \Km/\pim.

На основе анализа инвариантных \pt \ и \mt \ спектров были получены значения температуры химического вымораживания $T_{0}$ и средней скорости коллективного потока частиц $\left< u_T \right>$ как функций от количества нуклонов-участников \Npart.
Величина $T_{0}\approx170$ МэВ и является постоянной относительно значений \Npart, в то время как величина \ut \ увеличивается с увеличением значений \Npart. Данный результат может свидетельствовать о том, что при столкновениях, характеризующихся большими значениями $\left<N_{part}\right>$ (центральные Cu+Au, Au+Au, U+U столкновения), коллективные эффекты выражены сильнее, чем при столкновениях с малыми значениями $N_{part}$ (\pal, \heau \ столкновения).

Сравнение идентифицированных факторов ядерной модификации заряженных адронов показало, значения \rab, измеренные в системах с разной геометрией (\dau, \heau, Cu+Au, Au+Au и U+U) совпадают при одинаковых значениях \Npart.
Сделан вывод, что рождение идентифицированных заряженных адронов не зависит от геометрии и размера системы столкновения и определяется величиной лишь размером области перекрытия ядер, характеризующегося величиной \Npart.

В центральных столкновениях \heau, Cu+Au, U+U был обнаружен эффект увеличенного выхода протонов и антипротонов, что было объяснено доминированием вклада процессов рекомбинации в образовние иднентифицируемых заряженных адронов в диапазоне малых и промежуточных поперечных импульсов ($p_{T}<3$ ГэВ/$c$). 
В \pal \ столкновениях, а также в периферических столкновениях \heau, Cu+Au, U+U эффект увеличенного выхода протонов и антипротонов не наблюдалтся, что было интерпретировано доминированием вклада процессов фрагментации в образовние иднентифицируемых заряженных адронов в диапазоне промежуточных поперечных импульсов (2 ГэВ/$c$ $<p_{T}<3$ ГэВ/$c$).

Полученные значения инвариантных спектров заряженных адронов могут быть использованы для уточнения параметров теоретических моделей, реализованных в пакетах прикладных программ, таких как  AMPT, HIJING, PHSD и др. В частности, для уточнения радиуса рекомбинации в рекомбинационных моделях, реализованных в таких программных пакетах как AMPT, PHSD.