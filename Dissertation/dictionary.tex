\chapter*{Словарь терминов}             % Заголовок
\addcontentsline{toc}{chapter}{Словарь терминов}  % Добавляем его в оглавление

\textbf{PHENIX} : Pioneering High Energy Nuclear Interaction eXperiment, Иноваци­онный эксперимент по ядерным взаимодействиям в области больших энергий

\textbf{$\sqrt{s_{NN}}$} : суммарная кинетическая энергия сталкивающихся ядер в системе центра масс в пересчете на один нуклон

\textbf{$p_T$} : поперечный импульс, $p_T = \sqrt{p_x^2 +p_y^2}$

\textbf{$m_T$} : поперечная масса, $m_T = \sqrt{p_T^2 +m^2}$

\textbf{$R_{AB}$} : фактор ядерной модификации

\textbf{$\pi^-/\pi^+$, $K^-/K^+$, $\bar{p}/p$, $K^+/\pi^+$, $K^-/\pi^-$, $\bar{p}/\pi^-$, $p/\pi^+$} : отношения инвариантных спектров заряженных адронов соответствующих типов

\textbf{$N_{coll}$} : среднее число парных неупругих нуклон-нуклонных взаимодействий в акте ядро-ядерного столкновения

\textbf{$N_{part}$} : среднее число нуклонов, участвовавших хотя бы в одном неупругом нуклон-нуклонном взаимодействии в акте ядро-ядерного столкновения

\textbf{КГП} : Кварк-глюонная плазма

\textbf{КХД} : Квантовая хромодинамика

\textbf{CNM} : Cold Nuclear Matter -- эффекты холодной ядерной материи.

\textbf{AMPTsm} : A Multi-Phase Transport model «string melting version», мо­дель AMPT в конфигурации «плавление струн»

\textbf{PYTHIA} : генератор событий, широко использующийся для моделирования столкновений релятивистских частиц



