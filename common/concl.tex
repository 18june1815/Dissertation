%% Согласно ГОСТ Р 7.0.11-2011:
%% 5.3.3 В заключении диссертации излагают итоги выполненного исследования, рекомендации, перспективы дальнейшей разработки темы.
%% 9.2.3 В заключении автореферата диссертации излагают итоги данного исследования, рекомендации и перспективы дальнейшей разработки темы.
\begin{enumerate}
\item Были измерены инвариантные спектры по поперечному импульсу, факторы ядернонй модификации и отношения выходов идентифицируемых заряженных адронов ($\pi^\pm$, $K^\pm$, $p$, $\bar{p}$).

\item На основе анализа инвариантных \pt \ и \mt \ спектров были получены значения температуры химического вымораживания $T_{0}$ и средней скорости коллективного потока частиц $\left< u_T \right>$ как функций от \Npart.
Величина $T_{0}\approx170$ МэВ и является постоянной относительно значений \Npart, в то время как величина \ut \ увеличивается с увеличением значений \Npart. Данный результат может свидетельствовать о том, что при столкновениях, характеризующихся большими значениями $N_{part}$ (центральные Cu+Au, Au+Au, U+U столкновения), коллективные эффекты выражены сильнее, чем при столкновениях с малыми значениями $N_{part}$ (\pal, \heau \ столкновения).

\item В столкновениях, характеризующихся большими значениями \Npart (\Npart$>10$) был обнаружен увеличенный вход протонов и антипротонов, проявляющийся в увеличенных по сравнению с $p+p$ столкновениями (где \Npart$=2$) значениях $p/\pi$, а также в значениях (анти)протонных фактроров ядерной модификации ($R_{AB}>1$ для \prot \ и \aprot) в диапазоне средних поперечных импульсов ($2<p_T<3$ ГэВ/$c$).
\begin{comment}
В столкновениях с малыми значениями \Npart (\pal, \heau, \dau столкновениях и периферийных столкновениях Cu+Au, U+U) отношения \prot/$\pi$ близки к значениям, измеренным в \pp-столкновениях ($(p/\pi)^{р+р}$). В столкновениях с большими \Npart (центральные столкновения Cu+Au, Au+Au и U+U) отношения \prot/$\pi$ достигают значения 0.8, что в 2,5 раза больше, чем $(p/\pi)^{ р+р}$. Таким образом, в столкновениях тяжелых ионов (\cuau, \auau, \uu) отношения \prot/$\pi$ сильно зависят от центральности, а при малых \heau и \dau столкновениях зависимость от центральности незначительна в пределах неопределенности. В \pal-столкновениях зависимость от центральности отсутствует и значения \prot/$\pi$ согласуются с таковыми, измеренными в \pp-столкновениях, в пределах погрешностей.
Отношения $K/\pi$ во всех рассмотренных системах также не показывают зависимости от центральности.
Наблюдаемое поведение отношений $p/\pi$ и $K/\pi$ можно описать в рамках рекомбинационных моделей адронизации КГП.
\end{comment}

\item В столкновениях, характеризующихся малыми значениями \Npart (\Npart$<10$) обнаружен статистически значимый увеличенный вход протонов и антипротонов обнаружен не был.

\item Сравнение идентифицированных факторов ядерной модификации заряженных адронов показало, значения \rab, измеренные в системах с разной геометрией (\dau, \heau, Cu+Au, Au+Au и U+U) совпадают при одинаковых значениях \Npart.
Сделан вывод, что рождение идентифицированных заряженных адронов не зависит от геометрии и размера системы столкновения и определяется величиной лишь размером области перекрытия ядер, характеризующегося величиной \Npart.

%Наклон $R_{AB}(p_T)$ в коллизиях \pal более пологий, чем в коллизиях \heau и \dau при тех же значениях \Npart.
%Значения \rab для протонов в \pal-столкновениях равны единице в диапазоне 1 ГэВ/с $< p_T <$3 ГэВ/с, а значения \rab для протонов, измеренные в \heau и \dau-столкновениях, в пределах погрешностей больше единицы. Это может свидетельствовать о том, что идентифицированные механизмы рождения заряженных адронов в \pal и в $d$/$^{3}$He+Au столкновениях имеют различия.
\end{enumerate}
