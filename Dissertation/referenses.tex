\addcontentsline{toc}{chapter}{Литература}
\begin{thebibliography}{500}

\bibitem{IPCC2007}
\emph{IPCC}: 2007. Climate change. The physical science basis. Contribution of Working Group I to the Fourth Assessment Report of the Intergovernmental Panel on Climate Change.

\bibitem{Dymnikov2005}
\emph{Дымников~В.П., Лыкосов~В.Н., Володин~Е.М., Галин~В.Я., Глазунов~А.В., Грицун~А.С., Дианский~Н.А., Толстых~М.А., Чавро~А.И.} Моделирование климата и его изменений. // «Современные проблемы вычислительной математики и математического моделирования», М.: Наука, 2005, Т.~2, С.~38--175.

\bibitem{VolodinDiansky2006}
\emph{Володин Е.М., Дианский Н.А.} Моделирование изменений климата в ХХ--ХХII столетиях с помощью модели общей циркуляции атмосферы и океана. // Изв. РАН. Физика атмосферы и океана, 2006. --- Том~42, №3. --- С.~291--306.

\bibitem{Katcov2008}
\emph{Катцов В.М., Мелешко В.П.} Современные приоритеты фундаментальных исследований климата. // Труды ГГО, 2008. --- вып.~557. --- С.~3--19.

\bibitem{RoecknerArpe1996}
\emph{Roeckner~E., Arpe~K., Bengtsson~L., Christoph~M., Claussen~M., Dümenil~L., Esch~M., Giorgetta~M., Schlese~U., Schulzweida~U.} The atmospheric general circulation model ECHAM-4: model description and simulation of present-day climate Max-Planck Institute for Meteorology, 1996, Report No.218, Hamburg, Germany, 90pp.
\end{thebibliography}
