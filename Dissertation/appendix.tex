
\chapter{Систематические неопределенности измерений выходов идентифицируемых заряженных адронов}\label{app:A} 

\begin{table}[h]
	\caption{Значения неопределенностей, связанных с исключением мертвых областей дрейфовой камеры}
	\label{table:systDCFiduciual}
	
	\begin{tabularx}{\linewidth}
	{ 
		| >{\raggedright\arraybackslash}X 
		| >{\centering\arraybackslash}X 
		| >{\centering\arraybackslash}X 
		| >{\centering\arraybackslash}X 
		| >{\centering\arraybackslash}X 
		| >{\centering\arraybackslash}X 
		| >{\centering\arraybackslash}X | }
	\hline
	Система &\pt (ГэВ/c) 
	&  0.5 -- 1.0 & 1.0 -- 1.5 & 1.5 -- 2.0 & 2.0 -- 3.0 &  3.0 -- 4.0  \\ \hline
	\multirow{6}{*}{\pal}
	&  \pip & 5 & 5.7 & 6.5 & -- & --    \\ \cline{2-7} 
	&  \pim & 5.4 & 6.1 & 6.8 & -- & --    \\ \cline{2-7} 
	&  \Kp & 4.2 & 4.6 & 4.9 &  -- & --   \\ \cline{2-7} 
	&  \Km & 4.5 & 5 & 5.6 & -- & --    \\ \cline{2-7} 
	&  \prot & 6.3 & 6.6 & 8.3 & 8.9 & 9.3    \\ \cline{2-7} 
	&  \aprot & 4.4 & 5.1 & 6.8 & 6.9 & 7.9    \\ \hline
	\multirow{6}{*}{\heau}
	&  \pip & 2.5  .9 & 2.1 & 3.7 & --    \\ \cline{2-7} 
	&  \pim & 10.2 & 9.9 & 8.8 & 6.7 & --    \\ \cline{2-7} 
	&  \Kp & 6.8 & 7.6 & 7.7 &  -- & --    \\ \cline{2-7} 
	&  \Km & 5.8 & 7.1 & 7.4 &  -- & --   \\ \cline{2-7} 
	&  \prot & 5.6 & 6.5 & 6.4 & 5.5 & 4    \\ \cline{2-7} 
	&  \aprot & 5.5 & 8.2 & 8.4 & 8.1 & 7.7    \\ \hline
	\multirow{6}{*}{Cu+Au}
	&  \pip & 6.4 & 6.7 & 6.1 & 5.2 & --    \\ \cline{2-7} 
	&  \pim & 1.9 & 5.8 & 7.1 & 5.7 & --    \\ \cline{2-7} 
	&  \Kp & 7.5 & 6.6 & 5.5 &  -- & --    \\ \cline{2-7} 
	&  \Km & 5.1 & 3.1 & 4.9 &  -- & --    \\ \cline{2-7} 
	&  \prot & 4.6 & 4.6 & 4 & 5.1 & 5    \\ \cline{2-7} 
	&  \aprot & 3.8 & 2.1 & 2.9 & 4.1 & 4    \\ \hline 
	\multirow{6}{*}{U+U}
	&  \pip & 10.5 & 10 & 8.8 & 7.5 & --    \\ \cline{2-7} 
	&  \pim & 3.6 & 5.9 & 6.5 & 5.4 & --    \\ \cline{2-7} 
	&  \Kp & 5.8 & 4.6 & 4.2 &  -- & --    \\ \cline{2-7} 
	&  \Km & 2.3 & 3 & 4.3 &  -- & --    \\ \cline{2-7} 
	&  \prot & 5.5 & 4.9 & 4 & 5.1 & --   \\ \cline{2-7} 
	&  \aprot & 4.2 & 2.1 & 3.4 & 4.2 & --  \\ \cline{2-7} 
	\hline
	\end{tabularx}
\end{table}

\begin{table}[h]
	\caption{Значения систематических неопределеннстей, связанных с ограничением $z$-координаты вер­шины столкновения при отборе событий.}
	\label{table:systDCZed}
	
	\begin{tabularx}{\linewidth}
	{ 
		| >{\raggedright\arraybackslash}X 
		| >{\centering\arraybackslash}X 
		| >{\centering\arraybackslash}X 
		| >{\centering\arraybackslash}X 
		| >{\centering\arraybackslash}X 
		| >{\centering\arraybackslash}X 
		| >{\centering\arraybackslash}X | }
	\hline
	&\pt (ГэВ/c) 
	&  0.5 -- 1.0 & 1.0 -- 1.5 & 1.5 -- 2.0 & 2.0 -- 3.0 &  3.0 -- 4.0  \\ \hline
	\multirow{6}{*}{\pal}
	& \pip & 7 & 5.7 & 5.9 & 5.9 & --    \\ \cline{2-7} 
	& \pim & 5.1 & 6.6 & 4.0 & 4.1 & --    \\ \cline{2-7} 
	& \Kp & 5.4 & 6.2 & 4.4 &  -- & --    \\ \cline{2-7} 
	& \Km & 4.3 & 6.2 & 6.1 &  -- & --    \\ \cline{2-7} 
	& \prot & 4.4 & 5.4 & 4.1 & 3 & 6.1    \\ \cline{2-7} 
	& \aprot & 2.8 & 1.9 & 2.5 & 3.9 & 4.3    \\ \hline
	\multirow{6}{*}{\heau}
	& \pip & 3.7 & 0.9 & 1.9 & 0.9 & --    \\ \cline{2-7} 
	& \pim & 4.1 & 2.9 & 2.6 & 2.7 & --    \\ \cline{2-7} 
	& \Kp & 2.2 & 1.4 & 3.3 &  -- & --    \\ \cline{2-7} 
	& \Km & 3.1 & 4.4 & 4.8 &  -- & --    \\ \cline{2-7} 
	& \prot & 3.2 & 1.7 & 2 & 2.7 & 4.5    \\ \cline{2-7} 
	& \aprot & 3.3 & 1.9 & 2.2 & 2.9 & 3.3    \\ \hline
	\multirow{6}{*}{Cu+Au}
	& \pip & 1.2 & 0.9 & 0.8 & 1 & --    \\ \cline{2-7} 
	& \pim & 7.9 & 7.4 & 5.7 & 5.9 & --    \\ \cline{2-7} 
	& \Kp & 5.1 & 3.4 & 2.5 &  -- & --    \\ \cline{2-7} 
	& \Km & 6.9 & 4.8 & 2.9 &  -- & --   \\ \cline{2-7} 
	& \prot & 3 & 3.7 & 3.1 & 4.3 & 4.3    \\ \cline{2-7} 
	& \aprot & 4.4 & 4.3 & 2.8 & 3.5 & 4.1   \\ \hline
	\multirow{6}{*}{U+U}
	& \pip & 8.4 & 6.6 & 4 & 3.1 & --    \\ \cline{2-7} 
	& \pim & 1.9 & 3.3 & 1.8 & 2.2 & --    \\ \cline{2-7} 
	& \Kp & 6.8 & 5.1 & 3.7 &  -- & --    \\ \cline{2-7} 
	& \Km & 5.9 & 3.6 & 1.4 &  -- & --   \\ \cline{2-7} 
	& \prot & 4.9 & 5.1 & 4.2 & 5.6 & --   \\ \cline{2-7} 
	& \aprot & 3.8 & 5.1 & 3 & 3.6 & --    \\ \cline{2-7}
	\hline
	\end{tabularx}
\end{table}



\begin{table}[h]
	\caption{Ограничение z-координаты вер­шины столкновения}
	\label{table:systTOFMatching}
	
	\begin{tabularx}{\linewidth}
		{ 
			| >{\raggedright\arraybackslash}X 
			| >{\centering\arraybackslash}X 
			| >{\centering\arraybackslash}X 
			| >{\centering\arraybackslash}X 
			| >{\centering\arraybackslash}X 
			| >{\centering\arraybackslash}X 
			| >{\centering\arraybackslash}X | }
		\hline
		&\pt (ГэВ/c) 
		&  0.5 -- 1.0 & 1.0 -- 1.5 & 1.5 -- 2.0 & 2.0 -- 3.0 &  3.0 -- 4.0  \\ \hline
		%TOF.E matching
		\multirow{6}{*}{\pal}
		& \pip & 3 & 5.9 & 6.2 & 6.1 & --    \\ \cline{2-7} 
		& \pim & 3.1 & 5.3 & 5.8 & 6.0 & --    \\ \cline{2-7} 
		& \Kp & 2.4 & 5.7 & 9.3 &  -- & --    \\ \cline{2-7} 
		& \Km & 2.8 & 4.8 & 9.3 &  -- & --   \\ \cline{2-7} 
		& \prot & 2.4 & 4.9 & 5.6 & 6.3 & 8.6    \\ \cline{2-7} 
		& \aprot & 2.7 & 2.9 & 3.6 & 4 & 4.9    \\ \hline
		\multirow{6}{*}{\heau}
		& \pip & 1.6 & 2.3 & 3.6 & 3.1 & --    \\ \cline{2-7} 
		& \pim & 1.2 & 3.1 & 3.7 & 3.3 & --    \\ \cline{2-7} 
		& \Kp & 2.5 & 2 & 3.5 &  -- & --   \\ \cline{2-7} 
		& \Km & 1.8 & 1.5 & 3.6 &  -- & --    \\ \cline{2-7} 
		& \prot & 1.2 & 1.4 & 1.9 & 2.1 & 3.1    \\ \cline{2-7} 
		& \aprot & 3.7 & 1 & 2.1 & 2.1 & 3.1    \\ \hline
		\multirow{6}{*}{Cu+Au}
		& \pip & 2.5 & 3.9 & 3.5 & 4 & --    \\ \cline{2-7} 
		& \pim & 0.8 & 1.8 & 2.3 & 1.2 & --    \\ \cline{2-7} 
		& \Kp & 2.7 & 1.4 & 2 &  -- & --    \\ \cline{2-7} 
		& \Km & 1.5 & 2.4 & 3.9 &  -- & --   \\ \cline{2-7} 
		& \prot & 2.3 & 2.8 & 3.8 & 4.4 & 4.5    \\ \cline{2-7} 
		& \aprot & 0.7 & 1.7 & 2.5 & 3.2 & 3.3    \\ \hline
		\multirow{6}{*}{U+U}
		& \pip & 6.6 & 8.3 & 7.9 & 8.1 & --   \\ \cline{2-7} 
		& \pim & 3.4 & 4.4 & 4.3 & 3.7 & --   \\ \cline{2-7} 
		& \Kp & 2.9 & 2.8 & 4.1 & --& --   \\ \cline{2-7} 
		& \Km & 3.3 & 4.8 & 5.1 & --& --   \\ \cline{2-7} 
		& \prot & 1.4 & 3.2 & 4.1 & 5 & --    \\ \cline{2-7} 
		& \aprot & 2.6 & 4.2 & 4.1 & 5.4 & --    \\ \cline{2-7} 
		\hline
	\end{tabularx}
\end{table}



\begin{table}[h]
	\caption{Неопределенность идентификации частиц во времяпролетной камере.}
	\label{table:systPID}
	
	\begin{tabularx}{\linewidth}
		{ 
			| >{\raggedright\arraybackslash}X 
			| >{\centering\arraybackslash}X 
			| >{\centering\arraybackslash}X 
			| >{\centering\arraybackslash}X 
			| >{\centering\arraybackslash}X 
			| >{\centering\arraybackslash}X 
			| >{\centering\arraybackslash}X | }
		\hline
		&\pt (ГэВ/c) 
		&  0.5 -- 1.0 & 1.0 -- 1.5 & 1.5 -- 2.0 & 2.0 -- 3.0 &  3.0 -- 4.0  \\ \hline
		\multirow{6}{*}{\pal}
		& \pip & 0.3 & 0.6 & 1.5 & --& --   \\ \cline{2-7} 
		& \pim & 0.6 & 1 & 2.1 & --& --   \\ \cline{2-7} 
		& \Kp & 0.6 & 1.1 & 6.9 & --& --   \\ \cline{2-7} 
		& \Km & 1.4 & 0.8 & 7.8 & --& --   \\ \cline{2-7} 
		& \prot & 2.7 & 0.8 & 1.4 & 2.8 & 6    \\ \cline{2-7} 
		& \aprot & 0.4 & 0.9 & 2.2 & 2.2 & 5    \\ \hline
		\multirow{6}{*}{\heau}
		& \pip & 0.6 & 0.9 & 1 & 3.3 & --   \\ \cline{2-7} 
		& \pim & 0.7 & 0.5 & 1.4 & 2.9 & --   \\ \cline{2-7} 
		& \Kp & 1.2 & 0.4 & 2.6 & --& --   \\ \cline{2-7} 
		& \Km & 1.5 & 0.4 & 3.3 & --& --   \\ \cline{2-7} 
		& \prot & 2.1 & 0.5 & 1.6 & 1.9 & 2.1    \\ \cline{2-7} 
		& \aprot & 1.6 & 1.6 & 1 & 0.9 & 2.4    \\ \hline
		\multirow{6}{*}{Cu+Au}
		& \pip & 6.7 & 6.9 & 7 & 6.7 & --   \\ \cline{2-7} 
		& \pim & 1 & 0.9 & 1.2 & 5.5 & --   \\ \cline{2-7} 
		& \Kp & 1.8 & 0.9 & 2.4 & --& --   \\ \cline{2-7} 
		& \Km & 3.7 & 3.1 & 4 & --& --   \\ \cline{2-7} 
		& \prot & 3 & 1.7 & 4.3 & 5.2 & 4.2    \\ \cline{2-7} 
		& \aprot & 4 & 1.2 & 1.3 & 1.6 & 2.1   \\ \hline
		\multirow{6}{*}{U+U}
		& \pip & 6.7 & 6.9 & 7 & 6.7 & --   \\ \cline{2-7} 
		& \pim & 0.9 & 1.4 & 3.7 & 15.1 & --   \\ \cline{2-7} 
		& \Kp & 2.1 & 1.6 & 3.1 & --& --   \\ \cline{2-7} 
		& \Km & 4.1 & 1.1 & 3.7 & --& --   \\ \cline{2-7} 
		& \prot & 3.1 & 0.8 & 3 & 2.7 & --   \\ \cline{2-7} 
		& \aprot & 5.5 & 4.1 & 3.2 & 3.9 & --   \\ \cline{2-7} 
		\hline
	\end{tabularx}
\end{table}
\newpage
\chapter{Значения параметров $T$ (ГэВ/$c^2$), вычисленные  идентифицируемых заряженных адронов в $p$+Al, $^3$He+Au, Cu+Au и U+U столкновениях}\label{app:B}
%\chapter{Значения}\label{app:B}
%\setlength{\midchapskip}{20pt}
\begin{table}[h]
	\caption{Значения параметров обратного наклона $T$ (ГэВ/$c^2$), вычисленные для положительно заряженных идентифицируемых адронов (\pip, \Kp, \prot) в столкновениях \pal, \heau, Cu+Au и U+U.}
	\label{table:Tinv_pos}
	
	\begin{tabularx}{\linewidth}
		{
			| >{\centering\arraybackslash}X
			| >{\centering\arraybackslash}X
			| >{\centering\arraybackslash}X
			| >{\centering\arraybackslash}X
			| >{\centering\arraybackslash}X | }
		\hline
		Система & \Npart     &  \pip & \Kp &\prot   \\ \hline
		%\bfseries{\pal}      &     &     &      \\
		\pal & 3.1 $\pm$ 0.1 &  178.88 $\pm$ 0.35  &  210.69 $\pm$ 0.77   &  --  \\
		&4.4 $\pm$ 0.3 &  183.83 $\pm$ 0.28  &  216.10 $\pm$ 1.22   &  -- \\
		&3.3 $\pm$ 0.1 &  178.20 $\pm$ 0.32  &  210.05 $\pm$ 1.43   &  --  \\
		&1.6 $\pm$ 0.2 &  173.88 $\pm$ 0.46  &  204.74 $\pm$ 1.16   &  --  \\
		\hline
		%\bfseries{\heau}       &     &     &      \\
		\heau & 11.3 $\pm$ 0.5  &  208.70 $\pm$ 0.07  &  235.84 $\pm$ 0.24  & 295.74 $\pm$ 0.20   \\
		&21.1 $\pm$ 1.3  &  214.33 $\pm$ 0.11  &  242.57 $\pm$ 0.38  & 309.27 $\pm$ 0.33    \\
		&15.4 $\pm$ 0.9  &  209.84 $\pm$ 0.13  &  237.27 $\pm$ 0.44  & 296.44 $\pm$ 0.37  \\
		&9.5 $\pm$ 0.6   &  202.71 $\pm$ 0.15  &  227.37 $\pm$ 0.53  & 280.01 $\pm$ 0.44    \\
		&4.8 $\pm$ 0.3   &  191.01 $\pm$ 0.18  &  213.44 $\pm$ 0.66  & 254.22 $\pm$ 0.55    \\
		\hline
		%\bfseries{Cu+Au}       &     &     &       \\
		Cu+Au&70.4 $\pm$ 3.0  &  191.72 $\pm$ 0.01 &  249.33 $\pm$ 0.03 &  363.65 $\pm$ 0.09     \\
		&154.8 $\pm$ 4.1 &  194.29 $\pm$ 0.12 &  251.43 $\pm$ 0.04 &  383.85 $\pm$ 0.13     \\
		&80.4 $\pm$ 3.3  &  192.07 $\pm$ 0.02 &  249.14 $\pm$ 0.06 &  357.54 $\pm$ 0.16     \\
		&34.9 $\pm$ 1.8  &  185.32 $\pm$ 0.03 &  238.05 $\pm$ 0.09 &  313.90 $\pm$ 0.20     \\
		&11.5 $\pm$ 1.8  &  175.24 $\pm$ 0.05 &  222.76 $\pm$ 0.18 &  270.86 $\pm$ 0.28     \\
		\hline
		%\bfseries{U+U}       &     &     &     \\
		U+U&330 $\pm$ 6 &  198.18 $\pm$ 0.02  &  266.49 $\pm$ 0.08  & 382.54 $\pm$ 0.19  \\
		&259 $\pm$ 7 &  197.28 $\pm$ 0.04  &  269.42 $\pm$ 0.12  & 358.24 $\pm$ 0.23  \\
		&65 $\pm$ 6  &  192.42 $\pm$ 0.06  &  259.69 $\pm$ 0.22  & 314.92 $\pm$ 0.32  \\
		\hline
	\end{tabularx}
\end{table}

\newpage
\begin{table}[]
	\caption{Значения параметров обратного наклона $T$ (ГэВ/$c^2$), вычисленные для отрицательно заряженных идентифицируемых адронов (\pim, \Km, \aprot) в столкновениях \pal, \heau, Cu+Au и U+U.}
	\label{table:Tinv_neg}
	
	\begin{tabularx}{\linewidth}
		{
			| >{\centering\arraybackslash}X
			| >{\centering\arraybackslash}X
			| >{\centering\arraybackslash}X
			| >{\centering\arraybackslash}X
			| >{\centering\arraybackslash}X | }
		\hline
		
		Система &\Npart     & \pim & \Km & \aprot     \\ \hline
		%\bfseries{\pal}      &     &     &     \\
		\pal&3.1 $\pm$ 0.1 &  187.36 $\pm$ 0.18  &  206.61 $\pm$ 0.78    &  269.51 $\pm$ 1.25    \\
		&4.4 $\pm$ 0.3 &  192.56 $\pm$ 0.29  &  211.25 $\pm$ 1.21 &  275.14 $\pm$ 1.16    \\
		&3.3 $\pm$ 0.1 &  186.96 $\pm$ 0.33  &  206.25 $\pm$ 1.46  &  266.62 $\pm$ 2.28    \\
		&1.6 $\pm$ 0.2 &  181.99 $\pm$ 0.26  &  201.47 $\pm$ 1.18  &  254.20 $\pm$ 1.82    \\
		\hline
		%\bfseries{\heau}       &     &     &      \\
		\heau&11.3 $\pm$ 0.5  &  187.25 $\pm$ 0.06  &  236.98 $\pm$ 0.25 &  303.59 $\pm$ 0.24    \\
		&21.1 $\pm$ 1.3  &  191.58 $\pm$ 0.09  &  242.91 $\pm$ 0.39  &  317.40 $\pm$ 0.76    \\
		&15.4 $\pm$ 0.9  &  188.33 $\pm$ 0.10  &  238.49 $\pm$ 0.46  &295.10 $\pm$ 0.18    \\
		&9.5 $\pm$ 0.6   &  182.66 $\pm$ 0.13  &  229.34 $\pm$ 0.55  &  287.12 $\pm$ 0.54    \\
		&4.8 $\pm$ 0.3   &  172.40 $\pm$ 0.15  &  215.99 $\pm$ 0.70  &  261.83 $\pm$ 0.66    \\
		\hline
		%\bfseries{Cu+Au}       &     &     &         \\
		Cu+Au&70.4 $\pm$ 3.0  &  205.41 $\pm$ 0.01  &  253.80 $\pm$ 0.03  & 344.75 $\pm$ 0.08    \\
		&154.8 $\pm$ 4.1 &  208.57 $\pm$ 0.01  &  255.85 $\pm$ 0.09  & 365.18 $\pm$ 0.13    \\
		&80.4 $\pm$ 3.3  &  205.67 $\pm$ 0.02  &  253.34 $\pm$ 0.05  & 338.92 $\pm$ 0.15    \\
		&34.9 $\pm$ 1.8  &  197.72 $\pm$ 0.03  &  243.35 $\pm$ 0.09  & 306.48 $\pm$ 0.20    \\
		&11.5 $\pm$ 1.8  &  186.06 $\pm$ 0.05  &  228.90 $\pm$ 0.18  & 263.38 $\pm$ 0.29    \\
		\hline
		%\bfseries{U+U}       &     &     &       \\
		U+U&330 $\pm$ 6 &  205.84 $\pm$ 0.21  &  265.80 $\pm$ 0.08  &  374.48 $\pm$ 0.21    \\
		&259 $\pm$ 7 &  202.39 $\pm$ 0.03  &  266.36 $\pm$ 0.11  &  353.94 $\pm$ 0.26    \\
		&65 $\pm$ 6  &  197.74 $\pm$ 0.06  &  257.80 $\pm$ 0.20  &  308.57 $\pm$ 0.35    \\
		\hline
	\end{tabularx}
\end{table}

\clearpage
\chapter{Значения аппроксимационных параметров \To \ и \ut, вычисленные для различных систем столкновений: \pal, \heau, Cu+Au, U+U.}\label{app:С}
\begin{table}[h]
	\caption{Значения аппроксимационных параметров \To \ и \ut, вычисленные для различных систем столкновений: \pal, \heau, Cu+Au, U+U.}
	\label{table:To_ut}
	
	\begin{tabularx}{\linewidth}
		{
			| >{\centering\arraybackslash}X
			| >{\centering\arraybackslash}X
			| >{\centering\arraybackslash}X
			| >{\centering\arraybackslash}X
			| >{\centering\arraybackslash}X | }
		\hline
		
		\Npart     & $T_{0}^{+}$ & $T_{0}^{-}$  & $\left<u_{t}\right>^{+}$ & $\left<u_{t}\right>^{-}$   \\ \hline
		
		\bfseries{\pal}       &     &     &      &    \\
		3.1$\pm$0.1  &  167.9 $\pm$ 2.1  &  166.4 $\pm$ 15.2  &  0.2 $\pm$ 0.1  &  0.3 $\pm$ 0.1   \\
		4.4$\pm$0.3   &  171 $\pm$ 0.9  &  171.3 $\pm$ 15.9  &  0.2 $\pm$ 0.1  &  0.3 $\pm$ 0.1 \\
		3.3$\pm$0.1  &  167.9 $\pm$ 2.1  &  167.8 $\pm$ 14.3  &  0.2 $\pm$ 0.1  &  0.3 $\pm$ 0.1    \\
		1.6$\pm$0.2  &  164 $\pm$ 4.3  &  163.8 $\pm$ 11.2  &  0.2 $\pm$ 0.1  &  0.3 $\pm$ 0.1    \\
		\hline
		\bfseries{\heau}       &     &     &      &    \\
		11.3$\pm$0.5  &  188.9 $\pm$ 10.2  &  166.2 $\pm$ 1.7  &  0.3 $\pm$ 0.1  &  0.3 $\pm$ 0.1    \\
		21.1$\pm$1.3  &  193 $\pm$ 12.4  &  166.4 $\pm$ 3.2  &  0.3 $\pm$ 0.1  &  0.3 $\pm$ 0.1    \\
		15.4$\pm$0.9  &  188.9 $\pm$ 9.8  &  167.5 $\pm$ 1.2  &  0.3 $\pm$ 0.1  &  0.3 $\pm$ 0.1    \\
		9.5$\pm$0.6  &  185 $\pm$ 8.5  &  164.6 $\pm$ 0.2  &  0.3 $\pm$ 0.1  &  0.3 $\pm$ 0.1    \\
		4.8$\pm$0.3  &  177 $\pm$ 4.9  &  158.3 $\pm$ 3.4  &  0.2 $\pm$ 0.1  &  0.3 $\pm$ 0.1    \\
		\hline
		\bfseries{Cu+Au}       &     &     &      &    \\
		70.4$\pm$3.0 &  153.9 $\pm$ 16.6  &  176 $\pm$ 11.9  &  0.4 $\pm$ 0.1  &  0.4 $\pm$ 0.1 \\
		154.8$\pm$4.1  &  150 $\pm$ 23.9  &  172.8 $\pm$ 19.7  &  0.4 $\pm$ 0.1  &  0.4 $\pm$ 0.1    \\
		80.4$\pm$3.3  &  157 $\pm$ 14.5  &  178.1 $\pm$ 10.2  &  0.4 $\pm$ 0.1  &  0.4 $\pm$ 0.1    \\
		34.9$\pm$2.9  &  160.9 $\pm$ 3.8  &  177.7 $\pm$ 2.3  &  0.4 $\pm$ 0.1  &  0.3 $\pm$ 0.1    \\
		11.5$\pm$1.8 &  159.9 $\pm$ 4.5  &  175.8 $\pm$ 7.5  &  0.3 $\pm$ 0.1  &  0.3 $\pm$ 0.1    \\
		\hline
		\bfseries{U+U}       &     &     &      &    \\
		330$\pm$6 &  159.9 $\pm$ 12  &  170.8 $\pm$ 13.2  &  0.4 $\pm$ 0.1  &  0.4 $\pm$ 0.1  \\
		259$\pm$7 &  168.9 $\pm$ 0.6  &  174.7 $\pm$ 2.9  &  0.4 $\pm$ 0.1  &  0.4 $\pm$ 0.1    \\
		65$\pm$6   &  175.9 $\pm$ 11.4  &  182.6 $\pm$ 9.6  &  0.3 $\pm$ 0.1  &  0.3 $\pm$ 0.1 \\
		\hline
	\end{tabularx}
\end{table}