{\actuality} Кварк-глюонная плазма (КГП) - это состояние вещества, которое существует при температуре $>175$ МэВ и плотности $\sim1$ ГэВ/Фм$^3$ \autocite{QGP} и состоит из асимптотически свободных сильно взаимодействующих кварков и глюонов. Считается, что в первые доли секунды ($10^{-11}$ c) после Большого Взрыва вселенная находилась в состоянии КГП. Согласно квантовой хромодинамике, КГП может образовываться также и в столкновениях релятивистских тяжелых ионов. Теоретические предсказания были подтверждены наблюдением в столкновениях релятивистских тяжелых ионов таких эффектов как увеличенный выход странности \autocite{StrangEnh, Strangeness_QGP}, увеличенный выход барионов \autocite{BaryonPuzzleHeavy,p2piRatio_2003,p2piRatio_130GeV}, эффект гашения струй \autocite{JetQuenching1, JetQuenching2, JetQuenching3}, ненулевые значения эллиптических потоков частиц, подавление выходов тяжелых кваркониев. 


В столкновениях релятивистских тяжелых ионов (таких как Cu+Au, Au+Au, U+U) температура образовавшейся КГП составляет $T\gtrsim300$ МэВ \autocite{Coalescence_models}. По мере расширения, КГП остывает и при достижении критической температуры ($\sim175$ МэВ) происходит процесс адронизации \cite{QGP, QGP, Coalescence_models} - фазовый переход в нейтральную по цвету адронную материю \autocite{nucleus2020}. 

Основными моделями адронизации являются модели фрагментации \cite{FragmentationLund} и рекомбинации \autocite{Coalescence_models, Recombination1, Recombination2}. Модель фрагментации основана на принципе конфайнмента: адроны формируются в результате разры­ва струн между сильновзаимодействующими высокоэнергетичными партонами.
Согласно рекомбинационной модели адроны образуются в результате объединения кварков, находящихся в конечном объеме фазового пространства, определяемого радиусом рекомбинации. 
В настоящее время предполагается, что рекомбинационные процессы преобладают в области $p_T \lesssim 2$ ГэВ/$c$, в то время как при $p_T \gtrsim 4$ ГэВ/$c$ преобладают
процессы фрагментации. В диапазоне промежуточных поперечных импульсов (2 ГэВ/$c \lesssim p_T \lesssim 4$ ГэВ/$c$) процессы фрагментации и рекомбинации являют­ся конкурирующими.
Роль рекомбинационных процессов возрастает в столкновениях релятивистских тяжелых ионов (A+B) по сравнению с протон-протонными столкновениями, что принято связывать \cite{Recombination1, Recombination2, BaryonPuzzleHeavy} с формированием КГП в A+B столкновениях.
Образование КГП увеличивает вероятность кварков оказаться в конечном объеме фазового пространства, определяемого радиусом рекомбинации, а также приводит к потерям энергии высокоэнергетичных партонов в сильных взаимодействиях с горячей и плотной средой.

Поскольку процессы адронизации являются непертурбативными в рамках квантовой хромодинамики, их теоретическое описание является чрезвычайно сложной задачей. В связи с этим особую важность приобретает анализ экспериментальных данных \autocite{nucleus2020}. Увеличение вклада рекомбинационных процессов может быть исследовано с помощью измерения выходов идентифицируемых заряженных адронов и отношения выходов (анти)протонов к выходам  \pipm-мезонов ($p/\pi$). 
В протон-протонных столкновениях в области поперечных импульсов 2 ГэВ/$c$ $<p_{T}<4$ ГэВ/$c$ барионов рождается в $\sim3$ раза меньше, чем мезонов. Это связано с большими массами барионов и требованием ненулевого барионного числа для образования бариона. Однако экспериментом PHENIX \cite{PHENIXoverview} было обнаружено, что в центральных Au+Au столкновениях величина $p/\pi$ достигает значения 0.8 \cite{p2piRatio_130GeV, p2piRatio_2003}. С уменьшением центральности столкновений различие между результатами в протон-протонных взаимодействиях и Au+Au взаимодействиях уменьшается.  
Данный эффект увеличения выхода протонов в столкновениях релятивистских ионов по сравнению с $p+p$ столкновениями был объяснен в рамках модели рекомбинации и считается одним из признаков образования КГП \cite{BaryonPuzzleHeavy, Recombination1, Recombination2}.

Долгое время считалось \cite{PHENIX_Nature,CNM}, в легких системах столкновений, таких как таких как \pal, \heau, \dau \ условия, необходимые для образования КГП, не достигаются, а процессы рождения частиц обуславливают лишь эффекты холодной ядерной материи \cite{CNM, phi_dAu, QGP_small_syst}. К эффектам холодной ядерной материи относят эффекты начального состояния, такие как эффект Кронина \cite{Cronin, Cronin_hadrons_pp_dAu_AuAu}, многократное рассеяние частиц \cite{MPI1, MPI2}, эффекты изоспина, ядерную модификацию партонных функций  распределения \cite{PDF1, PDF2} и т.п. Однако в 2018 г экспериментом PHENIX были обнаружены эллиптические потоки заряженных частиц в \pal, \heau, \dau \ столкновениях, интерпретированные как признак образования КГП \cite{PHENIX_Nature}. Данное исследование послужило толчком для дальнейшего изучения возможностей образования КГП в легких системах. 

Систематическое изучение механизмов адронизации и рекомбинационных процессов в легких (\pal, \heau) и тяжелых (Cu+Au, U+U) системах столкновений путем измерения выходов идентифицируемых заряженных адронов может быть использовано для определения минимальных условий образования КГП. 
 
Настоящая работа посвящена изучению особенностей рождения идентифицируемых заряженных адрнов (\pipm, \Kpm, \prot, \aprot) в столкновениях \pal, \heau, Cu+Au при энергии \sqsn \ = 200 ГэВ и U+U столкновениях при энергии \sqsn \ = 193 ГэВ.

\begin{comment}
\ifsynopsis
Этот абзац появляется только в~автореферате.
Для формирования блоков, которые будут обрабатываться только в~автореферате,
заведена проверка условия \verb!\!\verb!ifsynopsis!.
Значение условия задаётся в~основном файле документа (\verb!synopsis.tex! для
автореферата).
\else
Этот абзац появляется только в~диссертации.
Через проверку условия \verb!\!\verb!ifsynopsis!, задаваемого в~основном файле
документа (\verb!dissertation.tex! для диссертации), можно сделать новую
команду, обеспечивающую появление цитаты в~диссертации, но~не~в~автореферате.
\fi
\end{comment}
% {\progress}
% Этот раздел должен быть отдельным структурным элементом по
% ГОСТ, но он, как правило, включается в описание актуальности
% темы. Нужен он отдельным структурынм элемементом или нет ---
% смотрите другие диссертации вашего совета, скорее всего не нужен.

{\aim} данной работы является изучение процессов адронизации в столкновениях релятивистских ионов и минимальных условий формирования кварк-глюонной плазмы путем измерения заряженных адронов в столкновениях \pal, \heau, \cuau\, при энергии \sqsn=200 ГэВ и в столкновениях \uu \ при энергии \sqsn=193 ГэВ.

Для~достижения поставленной цели необходимо было решить следующие {\tasks}:
\begin{enumerate}[beginpenalty=10000] % https://tex.stackexchange.com/a/476052/104425
 \item Измерить инвариантные спектры по поперечному импульсу для \pipm, \Kpm, \prot \ и \aprot \ в столкновениях \pal, $^{3}$He+Au, Cu+Au при энергии $\sqrt{s_{NN}}$=200 ГэВ и в столкновениях U+U при энергии $\sqrt{s_{NN}}$=193 ГэВ.
\item Измерить факторы ядерной модификации для \pipm, \Kpm, \prot \ и \aprot \ в столкновениях  \pal, $^{3}$He+Au, Cu+Au при энергии $\sqrt{s_{NN}}$=200 ГэВ и в столкновениях U+U при энергии $\sqrt{s_{NN}}$=193 ГэВ.

\item Измерить отношения выходов \pim/\pip, \Km/\Kp, \prot/\aprot, \prot/\pip, \aprot/\pim, \Kp/\pip, \Km/\pim \ в столкновениях  \pal, $^{3}$He+Au, Cu+Au при энергии $\sqrt{s_{NN}}$=200 ГэВ и в столкновениях U+U при энергии $\sqrt{s_{NN}}$=193 ГэВ.
\item Провести физическую интерпретацию результатов.
\end{enumerate}


{\novelty}
\begin{enumerate}[beginpenalty=10000] % https://tex.stackexchange.com/a/476052/104425
	\item Впервые измерены инвариантные спектры рождения по поперечному импульсу заряженных адронов (\pipm, \Kpm, \prot, \aprot) в столкновениях  \pal, \heau, Cu+Au при энергии \sqsn=200 ГэВ и в столкновениях U+U при энергии \sqsn=193 ГэВ.
	\item Впервые получены факторы ядерной модификации для \pipm, \Kpm, \prot, \aprot\, в столкновениях  \pal, \heau, Cu+Au при \sqsn=200 ГэВ и в столкновениях U+U при \sqsn=193 ГэВ
	\item 	Выпервые измерены отношения выходов \pim/\pip, \Km/\Kp, \prot/\aprot, \prot/\pip, \aprot/\pim, \Kp/\pip, \Km/\pim\,  в столкновениях  \pal, \heau, Cu+Au при энергии \sqsn=200 ГэВ и в столкновениях U+U при энергии \sqsn=193 ГэВ.
\end{enumerate}

{\influence}
\begin{enumerate}[beginpenalty=10000] % https://tex.stackexchange.com/a/476052/104425
	\item Полученные значения инвариантных спектров заряженных адронов могут быть использованы для уточнения параметров теоретических моделей, реализованных в пакетах прикладных программ, таких как  AMPT, HIJING, PHSD и др. В частности, для уточнения радиуса рекомбинации в рекомбинационных моделях, реализованных в таких программных пакетах как AMPT, PHSD.
	\item Методика измерения выходов заряженных адронов, представленная в данной работе, может быть применена в аналогичных исследованиях таких экспериментов, как SPD и MPD.
	
\end{enumerate}


%{\methods} \ldots

{\defpositions}
\begin{enumerate}[beginpenalty=10000] % https://tex.stackexchange.com/a/476052/104425
	\item Впервые измеренные инвариантные спектры по поперечному импульсу, факторы ядерной модификации идентифицируемых заряженных адронов (\pipm, \Kpm, \prot, \aprot), а также величины отношений выходов \pim/\pip, \Km/\Kp, \prot/\aprot, \prot/\pip, \aprot/\pim, \Kp/\pip, \Km/\pim \ в столкновениях  \pal, \heau, Cu+Au при энергии \sqsn = 200 ГэВ и в столкновениях U+U при энергии \sqsn=193 ГэВ. 

	\item Особенности рождения \pipm, \Kpm, \prot \ или \aprot \ в \heau, Cu+Au, U+U  столкновениях не зависят от геометрии области перекрытия сталкивающихся ядер, а определяются количеством нуклонов-участников.
	
	\item В центральных столкновениях \heau, Cu+Au, U+U наблюдается эффект увеличенного выхода протонов и антипротонов, что может быть объяснено доминированием вклада процессов рекомбинации в образовние иднентифицируемых заряженных адронов в диапазоне малых и промежуточных поперечных импульсов ($p_{T}<4$ ГэВ/$c$). 
	
	\item В \pal \ столкновениях, а также в периферических столкновениях \heau, Cu+Au, U+U эффект увеличенного выхода протонов и антипротонов не наблюдается, что может быть объяснено доминированием вклада процессов фрагментации в образовние иднентифицируемых заряженных адронов в диапазоне промежуточных поперечных импульсов (2 ГэВ/$c$ $<p_{T}<4$ ГэВ/$c$).
	
	\item Значения температуры химического вымораживания (\To) и скоростей коллективного потока частиц (\ut), измеренные как функция от количества нуклонов- участников (\Npart) в \pal, \heau, Cu+Au и U+U столкновениях.
\end{enumerate}

{\reliability} Достоверность полученных результатов и методики измерений были подтверждены внутренней независимой проверкой коллаборации PHENIX. Результаты обсуждались на семинарах коллаборации PHENIX, а также международных конференциях. Результаты опубликованы в рецензируемых журналах баз ВАК и SCOPUS/WebOfScience.


{\probation}
Основные результаты работы докладывались~на международных конференциях:
\begin{enumerate}[beginpenalty=10000] % https://tex.stackexchange.com/a/476052/104425
	\item «Ядро-2020» (Дубна, РФ)
	\item «ICPPA-2020», (Москва, РФ)	
	\item «PhysicA.SPb 2020» (Санкт-Петербург, РФ)
	\item «Ядро-2021» (Санкт-Петербург, РФ)
	\item «PhysicA.SPb 2021» (Санкт-Петербург, РФ)
	\item «Lomonosov Conference» 2021 (Москва, РФ)
	\item «PhysicA.SPb 2022» (Санкт-Петербург, РФ)
	\item «ICPPA-2022» (Москва, РФ)
	\item «ICNFP-2022» (Крит, Греция)
	\item Конференция имени Б. С. Ишханова "Концентрированные потоки энергии в космической технике, электронике, экологии и медицине" (Москва, РФ)
\end{enumerate}
Результаты опубликованы в рецензируемых журналах баз ВАК и SCOPUS/WebOfScience, таких как <<Вестник Московского Университета (Физика)>>, <<Физика элементарных частиц и атомного ядра>>, <<Journal of Physics: Conference Series>>.	

{\contribution} Автор внес определяющий вклад в работу по отбору и анализу данных, получению физического результата и оценке систематических неопределенностей измерений.

\ifnumequal{\value{bibliosel}}{0}
{%%% Встроенная реализация с загрузкой файла через движок bibtex8. (При желании, внутри можно использовать обычные ссылки, наподобие ``).
    {\publications} Основные результаты по теме диссертации изложены
    в~XX~печатных изданиях,
    X из которых изданы в журналах, рекомендованных ВАК,
    X "--- в тезисах докладов.
}%
{%%% Реализация пакетом biblatex через движок biber
    \begin{refsection}[bl-author, bl-registered]
        % Это refsection=1.
        % Процитированные здесь работы:
        %  * подсчитываются, для автоматического составления фразы "Основные результаты ..."
        %  * попадают в авторскую библиографию, при usefootcite==0 и стиле `\insertbiblioauthor` или `\insertbiblioauthorgrouped`
        %  * нумеруются там в зависимости от порядка команд `\printbibliography` в этом разделе.
        %  * при использовании `\insertbiblioauthorgrouped`, порядок команд `\printbibliography` в нём должен быть тем же (см. biblio/biblatex.tex)
        %
        % Невидимый библиографический список для подсчёта количества публикаций:
        \printbibliography[heading=nobibheading, section=1, env=countauthor,          keyword=biblioauthorvak]%
        \printbibliography[heading=nobibheading, section=1, env=countauthorwos,          keyword=biblioauthorwos]%
        \printbibliography[heading=nobibheading, section=1, env=countauthorscopus,       keyword=biblioauthorscopus]%
        \printbibliography[heading=nobibheading, section=1, env=countauthorconf,         keyword=biblioauthorconf]%
        \printbibliography[heading=nobibheading, section=1, env=countauthorother,        keyword=biblioauthorother]%
        \printbibliography[heading=nobibheading, section=1, env=countregistered,         keyword=biblioregistered]%
        \printbibliography[heading=nobibheading, section=1, env=countauthorpatent,       keyword=biblioauthorpatent]%
        \printbibliography[heading=nobibheading, section=1, env=countauthorprogram,      keyword=biblioauthorprogram]%
        \printbibliography[heading=nobibheading, section=1, env=countauthor,             keyword=biblioauthor]%
        \printbibliography[heading=nobibheading, section=1, env=countauthorvakscopuswos, filter=vakscopuswos]%
        \printbibliography[heading=nobibheading, section=1, env=countauthorscopuswos,    filter=scopuswos]%
        %
        %\nocite{confbib1}
        %
        {\publications} Основные результаты по теме диссертации изложены в~\arabic{citeauthor}~печатных изданиях,
        \arabic{citeauthorvak} из которых изданы в журналах, рекомендованных ВАК\sloppy%
        \ifnum \value{citeauthorscopuswos}>0%
            , \arabic{citeauthorscopuswos} "--- в~периодических научных журналах, индексируемых Web of~Science и Scopus\sloppy%
        \fi%
        \ifnum \value{citeauthorconf}>0%
            , \arabic{citeauthorconf} "--- в~тезисах докладов.
        \else%
            .
        \fi%
        \ifnum \value{citeregistered}=1%
            \ifnum \value{citeauthorpatent}=1%
                Зарегистрирован \arabic{citeauthorpatent} патент.
            \fi%
            \ifnum \value{citeauthorprogram}=1%
                Зарегистрирована \arabic{citeauthorprogram} программа для ЭВМ.
            \fi%
        \fi%
        \ifnum \value{citeregistered}>1%
            Зарегистрированы\ %
            \ifnum \value{citeauthorpatent}>0%
            \formbytotal{citeauthorpatent}{патент}{}{а}{}\sloppy%
            \ifnum \value{citeauthorprogram}=0 . \else \ и~\fi%
            \fi%
            \ifnum \value{citeauthorprogram}>0%
            \formbytotal{citeauthorprogram}{программ}{а}{ы}{} для ЭВМ.
            \fi%
        \fi%
        % К публикациям, в которых излагаются основные научные результаты диссертации на соискание учёной
        % степени, в рецензируемых изданиях приравниваются патенты на изобретения, патенты (свидетельства) на
        % полезную модель, патенты на промышленный образец, патенты на селекционные достижения, свидетельства
        % на программу для электронных вычислительных машин, базу данных, топологию интегральных микросхем,
        % зарегистрированные в установленном порядке.(в ред. Постановления Правительства РФ от 21.04.2016 N 335)
    \end{refsection}%
    \begin{refsection}[bl-author, bl-registered]
        % Это refsection=2.
        % Процитированные здесь работы:
        %  * попадают в авторскую библиографию, при usefootcite==0 и стиле `\insertbiblioauthorimportant`.
        %  * ни на что не влияют в противном случае
        \nocite{physica2020}%vak
        \nocite{physica2021}%vak
        \nocite{icppa2020}%vak
        \nocite{lomcon2021}%vak
        \nocite{nucleus2020}%vak       
    \end{refsection}%
        %
        % Всё, что вне этих двух refsection, это refsection=0,
        %  * для диссертации - это нормальные ссылки, попадающие в обычную библиографию
        %  * для автореферата:
        %     * при usefootcite==0, ссылка корректно сработает только для источника из `external.bib`. Для своих работ --- напечатает "[0]" (и даже Warning не вылезет).
        %     * при usefootcite==1, ссылка сработает нормально. В авторской библиографии будут только процитированные в refsection=0 работы.
}

\begin{comment}
При использовании пакета \verb!biblatex! будут подсчитаны все работы, добавленные
в файл \verb!biblio/author.bib!. Для правильного подсчёта работ в~различных
системах цитирования требуется использовать поля:
\begin{itemize}
        \item \texttt{authorvak} если публикация индексирована ВАК,
        \item \texttt{authorscopus} если публикация индексирована Scopus,
        \item \texttt{authorwos} если публикация индексирована Web of Science,
        \item \texttt{authorconf} для докладов конференций,
        \item \texttt{authorpatent} для патентов,
        \item \texttt{authorprogram} для зарегистрированных программ для ЭВМ,
        \item \texttt{authorother} для других публикаций.
\end{itemize}
Для подсчёта используются счётчики:
\begin{itemize}
        \item \texttt{citeauthorvak} для работ, индексируемых ВАК,
        \item \texttt{citeauthorscopus} для работ, индексируемых Scopus,
        \item \texttt{citeauthorwos} для работ, индексируемых Web of Science,
        \item \texttt{citeauthorvakscopuswos} для работ, индексируемых одной из трёх баз,
        \item \texttt{citeauthorscopuswos} для работ, индексируемых Scopus или Web of~Science,
        \item \texttt{citeauthorconf} для докладов на конференциях,
        \item \texttt{citeauthorother} для остальных работ,
        \item \texttt{citeauthorpatent} для патентов,
        \item \texttt{citeauthorprogram} для зарегистрированных программ для ЭВМ,
        \item \texttt{citeauthor} для суммарного количества работ.
\end{itemize}
% Счётчик \texttt{citeexternal} используется для подсчёта процитированных публикаций;
% \texttt{citeregistered} "--- для подсчёта суммарного количества патентов и программ для ЭВМ.

Для добавления в список публикаций автора работ, которые не были процитированы в
автореферате, требуется их~перечислить с использованием команды \verb!\nocite! в
\verb!Synopsis/content.tex!.
\end{comment}