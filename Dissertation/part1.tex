\chapter{Название первой главы} \label{chapt1}

\section{Параграф номер один} \label{sect1_1}

Вот так может выглядеть формула: $\Delta Q$, которую необходимо вставить в строку по тексту.

А вот так выглядит ненумерованая отдельностоящая формула:
$$
\Delta R = \Delta Q + \lambda_f \Delta T_S,
$$
где $\lambda_f$ некоторая величина.


%============================================================================================================================

%\newpage
\section{Параграф номер два} \label{sect1_2_model}

Вот так можно написать две формулы, не нумеруя их, чтобы знаки равно были строго друг под другом:
\begin{eqnarray}\label{plasim25P1}
f_W & = & \min \left( 1, \max \left( 0, \frac{W_{soil} / W_{max}}{W_{crit}} \right)  \right), \nonumber \\
f_T & = & \min \left( 1, \max \left( 0, \frac{T_s / T_{melt}}{T_{crit}} \right)  \right), \nonumber
\end{eqnarray}

А вот так можно сослаться на приложение: подробное описание чего-нибудь содержится в Приложении \ref{AppendixA}.

%============================================================================================================================


%\newpage
\section{Параграф номер три} \label{sect1_2}

А вот так пишется нумерованая формула:

\begin{equation}\label{eq1}
\frac{d\textbf{X}}{dt}=F_e,
\end{equation}
где $\textbf{X}$~--- 
А вот это - ссылка на формулу (\ref{eq1})

%============================================================================================================================


%\newpage
\section{Параграф номер четыре}  \label{sect1_3}

Вот так можно вставить картинку:
\begin{figure} [htb] 
\centering
\includegraphics [scale=0.27] {CO2_Control_A2.eps}
\caption{Подпись к картинку.} 
\label{CO2}
\end{figure}


%============================================================================================================================

%\newpage
\section{Параграф номер пять} \label{sect1_4}

А это две картинки под общим номером и названием:
\begin{figure} [htbp] 
   \centering\begin{tabular}{c}
     \includegraphics [scale=0.25] {CO2_Control_A2.eps} \\ 
      а) \\
      $\:$ \\   
     \includegraphics [scale=0.25] {CO2_Control_A2.eps} \\
      б) \\
   \end{tabular}
   \caption{Подпись к рисунку.} 
   \label{FBtsCA2}
 \end{figure}


%============================================================================================================================

\newpage
\section{Параграф номер шесть} \label{sect1_5}

Содержание по пунктам (например, выводы)  можно отобразить вот так:
\begin{enumerate}
 \item Первый вывод.
 \item Следующий вывод.
 \item Очередной вывод.
\end{enumerate}


