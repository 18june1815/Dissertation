\chapter*{Список сокращений и условных обозначений} % Заголовок
\addcontentsline{toc}{chapter}{Список сокращений и условных обозначений}  % Добавляем его в оглавление
% при наличии уравнений в левой колонке значение параметра leftmargin приходится подбирать вручную
\begin{description}[align=right,leftmargin=3.5cm]
\item[%
    \(\begin{rcases}
        a_n\\
        b_n
    \end{rcases}\)%
    ] коэффициенты разложения Ми в дальнем поле соответствующие
электрическим и магнитным мультиполям
\item[%
    \({\boldsymbol{\hat{\mathrm e}}}\)%
    ] единичный вектор
\item[\(E_0\)] амплитуда падающего поля
\item[\(j\)] тип функции Бесселя
\item[\(k\)] волновой вектор падающей волны
\item[%
    \(\begin{rcases}
        a_n\\
        b_n
    \end{rcases}\)%
    ] и снова коэффициенты разложения Ми в дальнем поле соответствующие
электрическим и магнитным мультиполям. Добавлено много текста, так что описание группы условных
обозначений значительно превысило высоту этой группы...
\item[\(L\)] общее число слоёв
\item[\(l\)] номер слоя внутри стратифицированной сферы
\item[\(\lambda\)] длина волны электромагнитного излучения
в вакууме
\item[\(n\)] порядок мультиполя
\item[%
    \(\begin{rcases}
        {\mathbf{N}}_{e1n}^{(j)}&{\mathbf{N}}_{o1n}^{(j)}\\
        {\mathbf{M}_{o1n}^{(j)}}&{\mathbf{M}_{e1n}^{(j)}}
    \end{rcases}\)%
    ] сферические векторные гармоники
\item[\(\mu\)] магнитная проницаемость в вакууме
\item[\(r,\theta,\phi\)] полярные координаты
\item[\(\omega\)] частота падающей волны
\item[BEM] boundary element method, метод граничных элементов
\item[CST MWS] Computer Simulation Technology Microwave Studio программа для компьютерного
    моделирования уравнен Максвелла
\item[DDA] discrete dipole approximation, приближение дискретиных диполей
\item[FDFD] finite difference frequency domain, метод конечных разностей в~частотной области
\item[FDTD] finite difference time domain, метод конечных разностей во~временной области
\item[FEM] finite element method,  метод конечных элементов
\item[FIT] finite integration technique, метод конечных интегралов
\item[FMM] fast multipole method, быстрый метод многополюсника
\item[FVTD] finite volume time-domain, метод конечных объёмов во~временной области
\item[MLFMA] multilevel fast multipole algorithm, многоуровневый быстрый алгоритм многополюсника
\item[MoM] method of moments, метод моментов
\item[MSTM] multiple sphere T-Matrix, метод Т-матриц для множества сфер
\item[PSTD] pseudospectral time domain method, псевдоспектральный метод во~временной области
\item[TLM] transmission line matrix method, метод матриц линий передач
\end{description}
