\chapter{Методика работы} \label{chapt3}

\section{Критерии отбора данных} \label{sect3_cuts}
В данной работе были использованы данные, полученные экспериментом PHENIX, в столкновениях p+p (2005 год набора данных), Cu+Au и U+U (2012 г), \heau (2014г), p+Au и p+Al (2015 г).

\begin{table}
	\caption{Критерии отбора данных}
	\centerfloat{\begin{tabular}{|c|c|}
			\hline
			Vertex cut & $|z_{vtx}| < $ 30 cm \\ \hline
			Quality of the tracks & $31||63||51$ \\ \hline
			Zed of charged track & $|zed| < $ 75 cm \& $|zed|>$ 4 cm \\ \hline
			
			\multirow{4}{*}{Интервал \pt измерения \pipm} & p+Al - 0.5 ГэВ/с $< p_T < $ 2.0 ГэВ/с \\
			& \heau - 0.5 ГэВ/с $< p_T < $ 2.0 ГэВ/с  \\ 
			& \cuau - 0.5 ГэВ/с $< p_T < $ 2.0 ГэВ/с  \\ 
			& \uu - 0.5 ГэВ/с $< p_T < $ 2.0 ГэВ/с  \\ \hline
			\multirow{4}{*}{Интервал \pt измерения \Kpm} & p+Al - 0.5 ГэВ/с $< p_T < $ 2.0 ГэВ/с \\
			& \heau - 0.5 ГэВ/с $< p_T < $ 2.0 ГэВ/с  \\ 
			& \cuau - 0.5 ГэВ/с $< p_T < $ 2.0 ГэВ/с  \\ 
			& \uu - 0.5 ГэВ/с $< p_T < $ 2.0 ГэВ/с  \\ \hline
			\multirow{4}{*}{Интервал \pt измерения \prot  и \aprot} & p+Al - 0.5 ГэВ/с $< p_T < $ 2.0 ГэВ/с \\
			& \heau - 0.5 ГэВ/с $< p_T < $ 2.0 ГэВ/с  \\ 
			& \cuau - 0.5 ГэВ/с $< p_T < $ 2.0 ГэВ/с  \\ 
			& \uu - 0.5 ГэВ/с $< p_T < $ 2.0 ГэВ/с  \\ \hline
			
			
	\end{tabular}}
	\label{table:cuts}
\end{table}

Критерии отбора данных, использованные в анализе, приведены в таблице \ref{table:cuts}. Большинство приведенных критериев являются общепринятыми в эксперименте PHENIX. Так общепринятыми являются  выбор событий с минимальным отбором, ограничения z-координаты вершины столкновения и выбор бита качества отслеживания. Также в данной работе были использованы специфические критерии отбора данных -  критерии идентификации заряженных частиц с использованием дрейфовой камеры и времяпролетных детекторов. Критерии идентификации заряженных частиц рассмотрены в разделе \ref{sect3_PID}.

Приведенные критерии отбора данных были использованы и при Монте-Карло моделировании, которое проводилось для оценки эффективности регистрации частиц (п. \ref{sect3:EffRec}) и коррекции, учитывающей увеличение выхода протонов и антипротонов в результате распада частиц (п. \ref{sect3:FeedDown}). 

\section{Триггер минимального отбора}
В данной работе использовались события минимального отбора, которые определяются с помощью детекторов BBC.
События минимального отбора определяются следующими условиями: срабатывание двух или более ФЭУ в каждом ВВС счетчике и ограничение на z-координату вершины столкновения |z| < 75 см. Отбор данных событий осуществляется онлайн с помощью триггера BBC Level-1 (LVL1 триггер).

Эффективность триггера для ядерных взаимодействий с минимальным отбором, определяется с помощью Монте-Карло моделирования BBC. Отклик для всех 124 трубок ФЭУ и логика платы BBLL1 настраиваются при моделировании в соответствии с условиями эксперимента. Эффективность триггера для неупругих столкновений A+B составляет 92 ± 2\%. 

/* Может дальше и не нужно? Посмотреть у Жарко */
Чтобы отбросить небольшой процент триггеров BBC, являющихся «фоновыми» событиями, требуется совпадение ZDC хотя бы с одним нейтроном с обеих сторон. Доля триггеров BBC, которые также удовлетворяют условию ZDC (BBLL1 >= 2 \& ZDCNS)/(BBLL1 >= 2), показана на рис. 3.3 в зависимости от номера запуска. Соотношение имеет максимальное значение около 97,5\% (обозначено горизонтальной линией). Тот факт, что отношение падает до более низких значений в некоторых более поздних прогонах, согласуется с наблюдениями во время этих прогонов о том, что светимость была высокой, а триггер BBC имел более высокую фоновую частоту. Также возможно, что у BBC была «горячая» трубка в некоторых из этих прогонов. По консервативной оценке 2,5\% эксклюзивных триггеров BBC, 40\% этих событий связаны с неэффективностью ZDC, а 60\% являются «фоновыми» событиями. Эффективность триггера ZDC для событий, которые также удовлетворяют триггеру BBC, является триггером.

\section{Определение центральности} \label{sect3:centr}
Центральность является характеристикой столкновения, показывающей степень перекрытия ядер, и непосредственно связанной с прицельным параметром. Лобовым столкновениям с прицельным параметром равным нулю, в которых степень перекрытия ядер максимальна, соответствует значение центральности 0\%. Переферическим столкновениям, в которых ядра проходят по касательной, а прицельный параметр равен сумме радиусов сталкивающихся ядер, соответствует значения центральности 100 \%. В остальных случаях значения центральности варьируется от 0\% до 100 \%, характеризуя степень перекрытия ядер.


\begin{figure}[] 
	\centerfloat
	\includegraphics [width=0.7\linewidth]{Methodology/centrality.png}
	\caption{Распределение событий Cu+Au столкновений по величине заряда, зарегистрированного в счетчиках BBC} 
	\label{img:Met_centr}
\end{figure}

Центральность столкновения в эксперименте PHENIX определяется с помощью BBС детектора, который измеряет количество заряженных частиц в переднем и заднем диапазоне быстрот ($3<|\eta|<4$). 
Для определения центральности заполняется гистограмма распределения событий ядро-ядерных столкновений по величине заряда, зарегистрированного в BBC счетчике. Пример данной гистограммы для Cu+Au столкновений приведен на рис. \ref{img:Met_centr}. Далее полученная гистограмма разбивается на части, количество событий в которых пропорционально рассматриваемым интервалам центральности. Разбиение, соответствующее интервалам центральности 0–20 \%, 20–40 \%, 40–60 \%, 60–82 \% обозначено на графике линиями. Интервал центральности 60-92 \% используется как самы переферийный в связи с низкой статистикой в переферических столкновениях.



\begin{table}[]
	\caption{Значения \Ncoll \ и \Npart, вычисленные с помощью модели Глаубера для различных центральностей \pal, \heau, \cuau \ и \uu \ столкновений.}
	\label{table:NcollNpart}
	
	\begin{tabularx}{\linewidth}
		{
			| >{\raggedright\arraybackslash}X
			| >{\centering\arraybackslash}X
			| >{\centering\arraybackslash}X | }
		\hline
			Центральность & \Ncoll    &  \Npart       \\ \hline \hline
			        &       \bfseries{\pal}   &               \\  \hline
			0-72\%     & 2.1$\pm$0.2    & 3.1$\pm$0.1    \\  \hline
			0-20\%     & 3.4$\pm$0.3    & 4.4$\pm$0.3    \\ \hline
			20-40\%    & 2.3$\pm$0.2    & 3.3$\pm$0.1    \\ \hline
			40-72\%    & 1.7$\pm$0.1   & 1.6$\pm$0.2  \\ \hline \hline
			      &     \bfseries{\heau}      &             \\ \hline
			0-88\%     & 10.4$\pm$0.7 & 11.3$\pm$0.5    \\  \hline
			0-20\%     & 22.3$\pm$1.7 & 21.1$\pm$1.3    \\ \hline
			20-40\%    & 14.8$\pm$1.1 & 15.4$\pm$0.9    \\ \hline
			40-60\%    & 8.4$\pm$0.6  & 9.5$\pm$0.6     \\ \hline
			0-88\%    & 3.4$\pm$0.3  & 4.8$\pm$0.3     \\  \hline \hline
			       &     \bfseries{\cuau}      &             \\  \hline
			0-80\%     & 123.8$\pm$12.0  & 70.4$\pm$3.0    \\  \hline
			0-20\%     & 313.8$\pm$28.4  & 154.8$\pm$4.1     \\  \hline
			20-40\%    & 129.3$\pm$12.4  & 80.4$\pm$3.3     \\  \hline
			40-60\%    & 41.8$\pm$5.3    & 34.9$\pm$2.9   \\ \hline
			60-80\%    & 10.1$\pm$2.0    & 11.5$\pm$1.8   \\ \hline \hline
			       &     \bfseries{\uu}     &            \\ \hline
			0-20\%     & 935$\pm$98    & 330$\pm$6   \\ \hline
			20-40\%    & 335$\pm$33    & 259$\pm$7   \\ \hline
			40-60\%    & 81$\pm$13     & 65$\pm$6    \\ \hline
			60-80\%    & 17$\pm$4  & 18$\pm$3   \\
				\hline
	
	\end{tabularx}
\end{table}

Центральность столкновения связана с величинами $N_{part}$ и $N_{coll}$ - количеством нуклонов участников и количеством парных нуклон-нуклонных соударений соответственно. Величины $N_{part}$ и $N_{coll}$ были определены с помощью модели Глаубера и представлены в таблице \ref{table:NcollNpart}.

\section{Исключение проблемных сегментов} \label{sect3_DM}
За время экусплуатации эксперимента PHENIX некоторые сегменты дрейфовых камер вышли из строя. Для корректной оценки эффективности регистрации заряженных адронов необходимо исключить эти области из дальнейшего анализа. Методика исключения мертвых областей заключается в следующем.
Для каждой из дрейфовых камер, находящихся в восточном и западном плече эксперимента PHENIX составляется гистограмма загрузки, которая представляет собой распределение уровня загрузки дрейфовой камеры в зависимости от угла $\alpha$ и номера платы. ??? 

Как видно из рисунков \ref{img:Met_DMRun12}-\ref{img:Met_DMRun15}, на гистограммах загрузки дрейфовых камер присутствуют локальные области, уровень загрузки в которых значительно отличается от среднего камере. Такие области дрейфовых камер с пониженной загрузкой  принято называть "мертвыми" областями. Для уменьшения искажений мертвые области исключаются из анализа.
Совокупность мертвых областей, исключаемых из анализа, называется мертвой картой дрейфовой камеры.
На рис \ref{img:Met_DMRun12}-\ref{img:Met_DMRun15} представлены мертвые карты дрейфовых камер для наборов данных, полученных в 2012, 2014 и 2015 годах. Красным цветом обозначены границы мертвых областей, исключаемых из дальнейшего анализа.

\begin{figure}[] 
	\centerfloat
	\includegraphics [width=0.8\linewidth]{Methodology/DC_DM_HeAu.png}
	\caption{Гистограммах загрузки дрейфовых камер в 2012 году} 
	\label{img:Met_DMRun12}
\end{figure}

\begin{figure}[] 
	\centerfloat
	\includegraphics [width=0.8\linewidth]{Methodology/DC_DM_CuAu.png}
	\caption{Гистограммах загрузки дрейфовых камер в 2014 году} 
	\label{img:Met_DMRun14}
\end{figure}

\begin{figure}[] 
	\centerfloat
	\includegraphics [width=0.8\linewidth]{Methodology/DC_DM_pAl.png}
	\caption{Гистограммах загрузки дрейфовых камер в 2015 году} 
	\label{img:Met_DMRun15}
\end{figure}


\section{Измерение первичного выхода заряженных адронов} \label{sect3_PID}
Заряженные адроны регистрируются непосредственно с помощью времяпролетной и дрейфовых камер эксперимента PHENIX. Квадрат массы частиц, зарегистрированных во времяпролетной и дрейфовой камере, может быть определен в соответствии с выражением \ref{eq:TOFm2}:
\begin{equation}
	\label{eq:TOFm2}
	m^2 = \frac{p^2}{c^2} \left(\frac{t^2 c^2}{L^2} - 1 \right)
\end{equation}
где $p$ - импульс частицы, измеренный с помощью дрейфовой камеры, $L$ - длина, пройденная частицей в сцинтилляторной планке TOF, $c$ - скорость света. 

На рис. \ref{img:TOF_PID}а представлено распределение произведения квадрата массы и заряда регистрируемых адронов в зависимости от поперечного импульса. Сигналы, соответствующие протонам, каонам и пионам, на данном распределении хорошо различимы.
Диапазон поперечных импульсов разбивается на промежутки шириной 0.1 ГэВ/с, на каждом из которых сигналы заряженных адронов аппроксимируются функцией Гаусса. Пример такой аппроксимации в диапазоне поперечных импульсов 1.0-1.1 ГэВ/с представлен на рис. \ref{img:TOF_PID}. 
Далее зависимости от поперечного импульса полученных среднеквадратичных отклонений ($\Tilde{\sigma}_h$) и математических ожиданий ($\Tilde{m}^2_h$) функций Гаусса для адронов h (h=\pipm,\Kpm,\prot, \aprot ) аппроксимировались функцией  \ref{eq:TOFgaus_approx}.

\begin{equation}
	\label{eq:TOFgaus_approx}
	f(p_T) = P_0 +P_1/p_T + P_2/p_T^2 + P_3 \cdot exp(\sqrt{p_T}) +P_4 \cdot \sqrt{p_T} 
\end{equation}
где $P_i, i \in [1,4]$ - параметры аппроксимации.

Значения $\sigma_h$ и $m_h$, вычисленные по формуле \ref{eq:TOFgaus_approx}, использовались для идентификации частиц. Частица, с зарегистрированным квадратом массы $m^2$, идентифицируется как адрон h в том случае, если  $m^2$ удовлетворяет неравенствам $$ m^2_h -2\sigma_h < m^2 < m^2_h +2\sigma_h. $$
При значениях поперечных импульсов \pt~2-3 ГэВ/с происходит наложение сигналов заряженных адронов. В связи с этим было введено дополнительное условие, обеспечивающее непопадание массы частицы в диапазон $\pm 2\sigma_h$ соседних сигналов частиц.
Условия идентификации заряженных адронов приведены в таблице \ref{table:m2cuts}.


\begin{table}[]
	\caption{Условия идентификации частиц}
	\label{table:m2cuts}
	\begin{center}
		\begin{tabular}{|c|c|}
			\hline
			\multirow{2}{*}{\pipm} & $|m^2 - m_{\pi}|<2\sigma_{\pi}$ \\ 
			& $m^2 < m^2_{K}-2\sigma_{K}$ \\ \hline
			\multirow{3}{*}{\Kpm}  & $|m^2 - m_{K}|<2\sigma_{K}$  \\ 
			& $m^2 > m^2_{\pi}+2\sigma_{\pi}$\\  
			& $m^2 < m^2_{p}+2\sigma_{p}$ \\ \hline
			\multirow{2}{*}{\prot, \aprot} & $|m^2 - mean_{p}|<2\sigma_{p}$ \\ 
			& $m^2 > m^2_{K}+2\sigma_{K}$ \\ \hline
		\end{tabular}
	\end{center}
\end{table}



\begin{figure}[ht] 
	\centerfloat
	\includegraphics [width = 0.9\linewidth] {Methodology/TOF2.png}
	\caption{а) Двухмерное распределение произведения квадрата массы на заряд регистрируемых адронов в зависимости от поперечного импульса б) Пример аппроксимации сигналов от заряженных адронов функцией Гаусса в диапазоне поперечных импульсов 1.0 -1.1 ГэВ/с} 
	\label{img:TOF_PID}  
\end{figure}

\section{Оценка эффективности регистрации} \label{sect3:EffRec}

%To correct for geometrical acceptance, analysis cuts, particle interactions with detector materials, and in-flight decays (for pions and kaons), we use single particle Monte Carlo (MC) simulations. For these simulations, single particles are generated using a random generator, with flat distributions in rapidity, azimuth, and \pT. The random particles are then run through a geant simulation of PHENIX to determine the interactions of the single particles with the detector subsystems and support structures. Next, all the detector response information is fed through the usual PHENIX reconstruction software to produce simulated tracks. Finally, these simulated tracks are analyzed in the exact same way as tracks from the real data in order to determine the corrections. The total correction factor, FC(\pT), is given by the following relation:

%\begin{equation}
%  \label{eq:CorrFactor}
%    FC(p_T) = %\frac{dN_{output}/dp_T}{dN_{input}/dp_T} = %\epsilon_{acceptance}\epsilon_{efficiency}\epsilon{cuts}
%\end{equation}
%To correct for the detector occupancy effect, which is most significant in the TOFW, we run embedding simulations, where a track from single particle MC simulations is embedded into a real event, and the occupancy correction is determined from the relative efficiencies of reconstructing the single track in isolation and in the event. This correction is the largest in the most central Au+Au collisions where the multiplicity is the highest and therefore the occupancy effect is the strongest. In the most peripheral Au+Au collisions the multiplicity is low enough that there is essentially no effect. The same is true in d+Au collisions, where no correction is applied.

Для того чтобы оценить количество адронов, родившихся в вершине ядроядерных взаимодействий, измеренные значения их первичного выхода должны быть скорректированы на эффективность регистрации ($\epsilon_{reg}$) адронов, учитывающий ограниченный аксептанс времяпролетной камеры, детекторные эффекты (различные шумы, разрешение и калибровка, нелинейность, эффективность триггеров реального времени и т.д.) и используемые при отборе данных ограничения.

Оценка эффективности регистрации частиц была проведена с помощью пакета PISA (PHENIX Integrated Simulation Application). Данный пакет является программой моделирования методом Монте-Карло, разработанного для детектора PHENIX на основе GEANT3.
В проекте PISA смоделированы геометрия, пространственное расположение и материал изготовления детекторных подсистем спектрометра PHENIX, их пространственные, импульсные и энергетические разрешения, а также конфигурация магнитного поля, полностью соответствующие структуре реальной установки в данном цикле ядро-ядерных столкновений.

Частицы с заданными массами покоя, координатами вершины рождения, энергией генерируются с помощью генератора  случайных чисел с плоскими распределениями по быстроте, азимуту и поперечному импульсу \pt. Затем сгенерированные частицы поступают в качестве входных данных в программу PISA для определения взаимодействия отдельных частиц с подсистемами детектора PHENIX. 
%Информация, полученная в результате работы программы GEANT4 обрабатывается с помощью стандартного программного обеспечения эксперимента  PHENIX для реконструкции треков.Наконец, ...
Для обработки дfннах, полученных в результате моделирования, применяются те же критерии отбора событий и идентификации частиц, что и для экспериментальных данных.

Функция эффективности регистрации вычисляется согласно следующему соотношению:
$$\epsilon_{reg} = \frac{N_{reg}(p_T)}{N_{tot}(p_T)}$$
где $N_{reg}$ - количество адронов, зарегистрированных в Монте-Карло модели, $N_{tot}$ - общее количество адронов, разыгранное в Монте-Карло модели.

%Входными данными для проекта PISA являются выборки частиц со случайно разыгранными величинами характеристик: координатами вершины рождения, массой покоя, каналом распада, энергией и проекциями импульса, моделирующие спектр частиц, рожденных в процессе ядро-ядерных столкновений. Розыгрыш этих характеристик формируется в соответствии с плоскими распределениями по вершине вдоль оси $z$ поперечному импульсу, псевдобыстроте и азимутальному углу смоделированных частиц. 
В таблице \ref{tab:Met_sim} приведены числа событий и границы изменения величин вершины, поперечного импульса, псевдобыстроты и азимутального угла в смоделированных выборках для исследуемых адронов.

\begin{table}[h]
	\centering
	\caption{Параметры моделирования}
	\begin{tabular}{|c|c|c|c|c|c|c|}
		\hline
		Частица &
		\begin{tabular}[c]{@{}l@{}}Количество\\ событий\end{tabular} & 
		$\Delta$ y& 
		\begin{tabular}[c]{@{}l@{}}$\phi$\\ рад\end{tabular}& \begin{tabular}[c]{@{}l@{}}$p_T^{max}$\\ ГэВ/с\end{tabular} & \begin{tabular}[c]{@{}l@{}}$p_T^{min}$\\ ГэВ/с\end{tabular} & \begin{tabular}[c]{@{}l@{}}Вершина\\см\end{tabular} \\ 
		\hline \pip & $10 \cdot 10^6$ & -0.5 - 0.5 & $\pi/2 - 3\pi/2$ & 0.0 & 3.5 & -30 - 30 \\
		\hline \pim & $10 \cdot 10^6$ & -0.5 - 0.5 & $\pi/2 - 3\pi/2$ & 0.0 & 3.5 & -30 - 30 \\
		\hline \Kp & $10 \cdot 10^6$ & -0.5 - 0.5 & $\pi/2 - 3\pi/2$ & 0.0 & 2 & -30 - 30 \\
		\hline \Km & $10 \cdot 10^6$ & -0.5 - 0.5 & $\pi/2 - 3\pi/2$ & 0.0 & 2 & -30 - 30 \\
		\hline \prot & $10 \cdot 10^6$ & -0.5 - 0.5 & $\pi/2 - 3\pi/2$ & 0.0 & 5 & -30 - 30 \\
		\hline \aprot & $10 \cdot 10^6$ & -0.5 - 0.5 & $\pi/2 - 3\pi/2$ & 0.0 & 5 & -30 - 30 \\
		\hline
	\end{tabular}
	\label{tab:Met_sim}
\end{table}

Перед расчетом эффективности регистрации была выполнена проверка соответствия данных моделирования и экспериментальных данных путем сравнения распределений треков в данных выборках. Для DC-PC1 распределение углов $\alpha$ при моделировании было взвешено в соответствии с распределением экспериментальных данных, а затем нормировано для представления того же общего интеграла, что и в экспериментальных данных. Результаты сравнения распределений треков в DC-PC1 представлены на рисунке ?. Как можно видеть, результаты моделирования соответствуют результатам, полученным в эксперименте.

Аналогичная проверка была выполнена для восточного крыла времяпролетной камеры. Двумерные диаграммы распределения треков
на плоскость ToFE для данных, моделирования и проекций на оси показаны на рис. 3.2. Результаты моделирования также соответствуют результатам, полученным в эксперименте.

Различие между кривыми DCE и ToF.E учитывается при оценке систематических ошибок.

\begin{figure}[] 
	\centerfloat
	\includegraphics [width=0.9\linewidth]{Methodology/eff_hadron_pAl.png}
	\caption{Эффективность регистрации заряженных адронов в столкновениях \pal.} 
	\label{img:eff_pAl}
\end{figure}

\begin{figure}[] 
	\centerfloat
	\includegraphics [width=0.9\linewidth]{Methodology/eff_hadron_HeAu.png}
	\caption{Эффективность регистрации заряженных адронов в столкновениях \heau.} 
	\label{img:eff_HeAu}
\end{figure}

\begin{figure}[] 
	\centerfloat
	\includegraphics [width=0.9\linewidth]{Methodology/eff_hadron_CuAu.png}
	\caption{Эффективность регистрации заряженных адронов в столкновениях Cu+Au.} 
	\label{img:eff_CuAu}
\end{figure}

\begin{figure}[] 
	\centerfloat
	\includegraphics [width=0.9\linewidth]{Methodology/eff_hadron_UU.png}
	\caption{Эффективность регистрации заряженных адронов в столкновениях U+U.} 
	\label{img:eff_UU}
\end{figure}

%====================================================================================

\begin{figure}[] 
	\centerfloat
	\includegraphics [width=0.9\linewidth]{Methodology/TOF_proj_pAl.png}
	\caption{TOFproj \pal.} 
	\label{TOFproj_pAl}
\end{figure}

\begin{figure}[] 
	\centerfloat
	\includegraphics [width=0.9\linewidth]{Methodology/TOF_proj_HeAu.png}
	\caption{TOFproj  \heau.} 
	\label{img:TOFproj_HeAu}
\end{figure}

\begin{figure}[] 
	\centerfloat
	\includegraphics [width=0.9\linewidth]{Methodology/TOF_proj_CuAu.png}
	\caption{TOFproj  Cu+Au.} 
	\label{img:TOFproj_CuAu}
\end{figure}

\begin{figure}[] 
	\centerfloat
	\includegraphics [width=0.9\linewidth]{Methodology/TOF_proj_UU.png}
	\caption{TOFproj U+U.} 
	\label{img:TOFproj_UU}
\end{figure}

\section{Коррекция вклада слабых распадов} \label{sect3:FeedDown}
%Weak Decay Correction - поправка слабого взаимодействия
Для оценки количества протонов, родившихся непосредственно столкновении релятивистских ядер, была введена поправка $C_{FD}$, учитывающая образование протонов в результате слабых распадов. Поскольку наибольший вклад в образование протонов вносят $\Lambda$ - барионы, фактор $C_{FD}$ в данной работе был вычислен как роцента протонов, образовавшихся в результате распада $\Lambda$ барионов ($N_{\Lambda}^p$), от общей выборки протонов ($N_{tot}^p$ ):
\begin{equation}
	\label{eq:Lambda_pT}
	C_{FD} = \frac{N_{\Lambda}^p}{N_{tot}^p} 
\end{equation}

Для оценки величины $C_{FD}$ было проведено Монте-Карло моделирование рождения $\Lambda$ барионов, с теми же параметрами, которые были использованы для моделирования протонов. Для учета зависимости от поперечного импульса, гистограммы были взвешены с весом, вычисляемым по формуле \ref{eq:Lambda_pT}:
\begin{equation}
	\label{eq:Lambda_pT}
	W(p_{T}^{track}) = p_{T} e^{\frac{m_T-m_0}{T}}
\end{equation}
Данная форма выбрана для наиболее точного представления $\Lambda$ и спектра протонов.
Затем в эту долю вносят поправку на коэффициент ветвления и отношения выходов $\Lambda$ барионов к протонам. Были использованы следующие значения отношений $\Lambda/p$:

Измерения $\Lambda$ барионов автоматически учитывают распады $\Sigma^0$ барионов, поскольку $\Sigma^0$ барионы  электромагнитно распадаются на $\Lambda$ и фотоны с коэффициентом ветвления 100 \%. В данной работе не было учтено рождение протонов в других распадах, таких как распады заряженных состояний $\Sigma$-барионов, а также многостранных барионных состояний $\Xi$ и $\Omega$. 

Систематическая неопределенность процента протонов, образовавшихся в результате распада $\Lambda$ барионов, от общей выборки протонов оценивается в 25 \%, что в первую очередь связано с неопределенностью величины отношения $\Lambda/p$. Доля распадных протонов составляет порядка 10–40\%, поэтому изменение отношения $\Lambda /p$ на 25\% приводит к изменению спектра протонов примерно на 2.5–10\%.
На рис. \ref{img:FeedDown} показана зависимость доли распадных протонов и антипротонов от поперечного импульса. 

\begin{figure}[] 
	\centerfloat
	\includegraphics [width=0.7\linewidth]{Methodology/FeedDown.png}
	\caption{Значения $C_{FD}$, вычисленные в столкновениях p+Al, \heau, Cu+Au, U+U  столкновениях. Заштрихованные области обозначают систематические погрешности.} 
	\label{img:FeedDown}
\end{figure}

\section{Систематические неопределенности} \label{sect3:Syst}
Систематическая неопределенность учитывает возможную ошибку при произвольном выборе параметров анализа (например, выбор параметров аппроксимации пиков, выбор взвешивающей функции), либо ошибку, связанную с неидеальностью воспроизведения условий эксперимента в используемой Монте-Карло модели.
% В данном разделе приведены классификация систематических неопределенностей (п. \ref{sect3:SystClass}), описаны источники систематической неопределенности (п. \ref{sect3:SystSource}) и приведены результаты оценки систематической неопределенности (п. \ref{sect3:SystValues}).

%\subsection{Классификация неопределенностей измерений} \label{sect3:SystClass}
Измеряемые величины обладают статистической и систематической неопределенностью. Статистическая неопределенность измерения заряженных адронов обусловлена объемом используемой выборки. %Методика определения статистической неопределенности измерения выходов заряженных адронов приведена в п. \ref{sect3:SystValues}. По своей природе, величина статистической неопределенности не коррелирована по поперечному импульсу и центральности.
Систематические неопределенности измерений классифицированы по следующим группам. 
\begin{itemize}
	\item неопределенность «типа A» полностью не коррелирована по поперечному импульсу и центральности. 
	\item неопределенность «типа B» коррелирована по поперечному импульсу, однако точный закон корреляции не известен. К неопределенностям типа B относят все систематические неопределенности, которые не являются неопределенностями типов A и C. Для всех результатов измерений неопределенность типа A квадратично суммируется со статистической неопределенностью.
	\item неопределенность «типа C» полностью коррелированы по поперечному импульсу, т.е. сдвигают измеренные мезонные выходы в область больших либо меньших значений одинаково во всем диапазоне поперечного импульса. В данной работе неопределенность типа C обусловлена неопеределенностью измерения значений \Ncoll.
\end{itemize}

\subsection{Источники систематических неопределенностей измерения заряженных адронов} \label{sect3:SystSource}
Cистематическая неопределенность определения спектров рождения заряженных адронов может исходить от следующих величин: первичного выхода заряженных адронов, эффективности регистрации заряженных адронов, а также отношения величин первичного выхода и эффективности регистрации. Ниже представлено описание источников систематической неопределенности измерений, учитываемые в настоящей работе. 

\bfseries Неопределенности, связанные с исключением мертвых областей дрейфовой камеры.
\mdseries
Данная систематическая неопределенность связана с произвольностью определения границ мертвых областей исключаемых из дальнейшего рассмотрения. Для оценки величины такой неопределенности, было проведено сравнение значений первичных выходов заряженных адронов, измеренных стандартным способом, а также двумя альтернативными способами: без  исключения мертвых областей дрейфовой камеры, а также с исключением мертвых областей, уменьшенных на 5 мрад по сравнению со стандартным методом. Результаты приведены на рис ...


\bfseries Неопределеннсть, связанная с ограничением z-координаты при отборе событий. 
\mdseries
Для оценки Неопределеннсти, связанной с ограничением z-координаты при отборе событий проводится сравнение результатов измерения заряженных адронов со стандартным критерием отбора данных по ZED $|zed|<70$ и измерения заряженных адронов, с условием на zed $|zed|<40$ 

\bfseries Неопределенность идентификации частиц во времяпролетной камере.
\mdseries
вызвана погрешностью определения параметров функции \ref{eq:TOFgaus_approx} и значений  $\sigma_h$ и $m_h$ (п. \ref{sect3_PID}). Данная неопределенность была оценена путем сравнения первичных выходов заряженных адронов, идентифицированных согласно стандартным критериям, описанным в п \ref{sect3_PID}, а также идентифицированных с использованием условий $ m^2_h -2.5\sigma_h < m^2 < m^2_h +2.5\sigma_h $ и  $ m^2_h -1.5\sigma_h < m^2 < m^2_h +1.5\sigma_h $.

\bfseries Неопределенность, связанная с ограниченным аксептансом дрейфовой камеры
\mdseries
Систематическая неопределенность, связанная с неидеальностью воспроизведения геометрии электромагнитного калориметра в Монте-Карло модели, оценивалась по изменению отношения данных к моделирования (см. правую панель на рис. ...). Это изменение было получено путем выбора различных областей для нормировки результатов моделирования на представление того же
общего интеграла, что и в экспериментальных данных. Было выбрано 10 различных областей вдоль оси номеров плат ДК. Каждая область поочередно использовалась для нормировки распределений треков в DC-PC, полученных из экспериментальных данных и моделирования (правая панель на рис. 3.10). Для каждой из 10 проведенных нормировок были вычислены отношения полных интегралов распределений треков в DC-PC, полученных из экспериментальных данных и моделирования. Итоговое значение рассматриваемой погрешности было вычислено как среднеквадратичное значение отношения 10 полученных данных к интегралам моделирования. 

\subsection{Значения систематических неопределенностей спектров заряженных адронов} \label{sect3:SystValues}
Значения систематических неопределенностей измерения выхода заряженных адрнов в зависимости от их поперечного импульса со стороны различных источников неопределенности приведены в таблицах \ref{table:systDC}-\ref{table:systPID} приложения \label{AppendixA}. Основным источником систематической неопределенности измерения выхода заряженных адронов являются неопределенности, связанные с исключением мертвых областей дрейфовой камеры и идентификации частиц во времяпролетной камере. Соответствующая относительная величина систематической неопределенности увеличивается от 2 до 9\% с ростом поперечного импульса. 
Итоговые значения систематических погрешностей вычислялись как среднеквадратичное значение погрешностей различных типов. Полученные итоговые значения систематических неопределенностей приведены в таблице \ref{table:systTotal}.

\begin{table}[]
	\caption{Значения систематических погрешностей змерений заряженных адронов}
	\label{table:systTotal}
	
	\begin{tabularx}{\linewidth}
		{ 
			| >{\raggedright\arraybackslash}X 
			| >{\centering\arraybackslash}X 
			| >{\centering\arraybackslash}X 
			| >{\centering\arraybackslash}X 
			| >{\centering\arraybackslash}X 
			| >{\centering\arraybackslash}X 
			| >{\centering\arraybackslash}X | }
		\hline
		&\pt (ГэВ/c)     &  0.5 - 1 & 1 - 1.5 & 1.5 - 2 & 2.5 - 3 & 2.5 - 3     \\ \hline
		\multirow{6}{*}{p+Al}  
		&  \pip  &      9.7  &      10.5  &      11.3  &     -  &      -    \\ \cline{2-7}
		&  \pim  &      8.7  &      11  &      10.5  &      -  &      -    \\ \cline{2-7}
		&  \Kp  &      7.9  &      10.2  &      13.7  &      -  &      -    \\ \cline{2-7}
		&  \Km  &      7.7  &      9.9  &      15  &      -  &      -    \\ \cline{2-7}
		&  \prot  &      10.3  &      11.5  &      12.4  &      13.1  &      16.5    \\ \cline{2-7}
		&  \aprot  &      8.4  &      8.6  &      10.3  &      10.9  &      12.8    \\ \hline
		\multirow{6}{*}{\heau}
		&  \pip  &      5.7  &      4.2  &      5.7  &      6.7  &      -    \\ \cline{2-7}
		&  \pim  &      11.6  &      11.3  &      10.5  &      9.1  &      -    \\ \cline{2-7}
		&  \Kp  &      8.3  &      8.6  &      10  &      -  &      -    \\ \cline{2-7}
		&  \Km  &      8.7  &      9.9  &      11.3  &      -  &      -    \\ \cline{2-7}
		&  \prot  &      8.6  &      8.6  &      8.8  &      8.5  &      8.8    \\ \cline{2-7}
		&  \aprot  &      9.1  &      10  &      10.4  &      10.3  &      10.6    \\  \hline
		\multirow{6}{*}{Cu+Au}
		&  \pip  &      10.2  &      10.9  &      10.5  &      9.9  &      -    \\ \cline{2-7}
		&  \pim  &      8.9  &      10.1  &      10  &      10.5  &      -    \\ \cline{2-7}
		&  \Kp  &      10.2  &      8.3  &      7.5  &      -  &      -    \\ \cline{2-7}
		&  \Km  &      10  &      7.7  &      8.6  &      -  &      -    \\ \cline{2-7}
		&  \prot  &      9  &      9  &      9.7  &      11.3  &      -   \\ \cline{2-7}
		&  \aprot  &      9.3  &      7.9  &      7.7  &      8.8  &      -    \\  \hline
		\multirow{6}{*}{U+U}
		&  \pip  &      16.8  &      16.5  &      14.7  &      13.7  &      -    \\ \cline{2-7}
		&  \pim  &      6.2  &      8.8  &      9.4  &      16.9  &      -    \\ \cline{2-7}
		&  \Kp  &      10.2  &      8.3  &      8.3  &      -  &      -    \\ \cline{2-7}
		&  \Km  &      8.8  &      7.6  &      8.5  &      -  &      -    \\ \cline{2-7}
		&  \prot  &      10  &      9.8  &      9.8  &      11.2  &      -    \\ \cline{2-7}
		&  \aprot  &      10.3  &      10  &      9.1  &      10.5  &     -   \\  \hline
		
	\end{tabularx}
\end{table}

\section{Инвариантные спектры по поперечному импульсу} \label{sectRes_spectra}
Для понимания адронно-адронных столкновений высоких энергий Ферми предложил следующий статистический метод [7]. Из-за насыщения фазового пространства рождение множества частиц в результате элементарных столкновений высоких энергий согласуется с тепловым описанием [7, 8, 9]. В столкновениях тяжелых ионов можно ожидать гидродинамического поведения, то есть локального теплового равновесия и коллективного движения, из-за большого числа вторичных рассеяний. Адроны в конечном состоянии являются наиболее многочисленным и доминирующим источником информации о ранней стадии столкновений. Спектры импульсов адронов и плотности быстроты зависят от теплового замораживания и коллективного потока. Соотношения частиц чувствительны к химическим свойствам системы и механизму образования частиц. Недавний обзор существующих данных, полученных в основном из CERN-SPS, можно найти в литературе [10].

Измерение инвариантных спектров по поперечному импульсу является одним из основных инструментов для изучения рождения частиц в столкновениях релятивистских ионов.
Инвариантные спектры по поперечному импульсу вычисляются согласно формуле \ref{eq:pTspectra}:
\begin{equation}
	\label{eq:pTspectra}
	\frac{1}{2\pi p_T} \frac{d^2 N}{dp_T dy}=\frac{1}{2\pi p_T}\frac{N_h}{N_{evt} \varepsilon_{eff} \Delta p_T \Delta y}
\end{equation}
где $\Delta p_T$ – диапазон поперечных импульсов, $\Delta y$ – диапазон быстрот, $N_h$ - количество заряженных адронов $h$, зарегистрированных в диапазонах  $\Delta p_T, \Delta y$,  $\varepsilon_{eff}$ - эффективность регистрации адронов $h$.
Инвариантные спектры по поперечному импульсу связаны с распределением по инвариантной массе следующим образом:
\begin{equation}
	\label{eq:mTspectra}
	\begin{split}
		\frac{1}{2\pi p_T} \frac{d^2 N}{dp_T dy}=
		\frac{1}{2\pi \sqrt{m_T^2-m_0^2}} \frac{d^2 N}{d(\sqrt{m_T^2-m_0^2})dy}=\\
		\frac{1}{2\pi \sqrt{m_T^2-m_0^2}} \frac{d^2 N \sqrt{m_T^2-m_0^2}}{m_T dm_Tdy}=\frac{1}{2\pi m_T} \frac{d^2 N}{dm_Tdy}
	\end{split}
\end{equation}
где $m_0$ - масса адрона, $m_T = \sqrt{p_{T}^{2}+m_0^2}$.


Согласно моделям термодинамики[], в области $m_T<$1.5 ГэВ/с спектры имеют экспоненциальную форму и могут быть описаны  зависимостью \ref{eq:mTspectraEXP}:
\begin{equation}
	\label{eq:mTspectraEXP}
	\frac{1}{2\pi m_T} \frac{d^2 N}{dm_T dy}=\frac{1}{2\pi T (T+m_0)}\cdot A \cdot exp{\frac{-m_T -m_0}{T}}
\end{equation}
$T$ - параметр обратного наклона спектра, $A$ - нормировочный коэффициент, содержащий информацию о множественности рождения частиц в данном диапазоне быстрот $dN/dy$.
%В области \mt$>2$ ГэВ/с спектры начинают проявлять степенную зависимость от \mt. 

Параметр обратного наклона $T$ пропорционален средней поперечной кинетической энергии, следовательно, в рамках термодинамических моделей, проявляет следующую зависимость от массы адрона:
$$ T \sim T_{thermal}+m_0 \cdot \langle u_T ^2 \rangle$$

%dthesis_CH_PHENIX

/* Collective expansion */
Наиболее удачное описание различных параметров наклона и изменения формы, наблюдаемых в $m_T$-спектрах в $A+A$-столкновениях, дает модель, включающая общее поперечное расширяющееся поле скоростей вместе с умеренной температурой термализованной системы. Качественно зависимость параметров обратного наклона от массы при наличии поперечного поля скоростей можно понять следующим образом [12]. В случае чисто теплового движения все частицы (независимо от их массы) двигались бы с одной и той же средней кинетической энергией, определяемой температурой, т. е.

\begin{equation}
	\left< E_{kin} \right>  \sim T_{thermal} 
\end{equation}

С другой стороны, при чисто коллективном движении все частицы двигались бы с одинаковой скоростью $\beta_T$ и, следовательно, средняя кинетическая энергия возрастала бы пропорционально их массе $m_0$, так как

\begin{equation}
	\left< E_{collective} \right>  \sim \frac{m_0\beta_T^2}{2}
\end{equation}

В предположении полного разделения между тепловым и коллективным движением частицы суперпозиция обоих типов движения даст массовую зависимость;

\begin{equation}
	\begin{split}
		\left< E_{kin} \right> = \left< E_{thermal} \right> + \left< E_{collective} \right> = T_{thermal} + \frac{m_0 \left< \beta_T \right>^2}{2} 
	\end{split}
\end{equation}

где $\left< \beta_T \right>$ — усредненная коллективная скорость для всех видов частиц. Параметр обратного наклона $T_0$ пропорционален средней поперечной кинетической энергии и определяется как

\begin{equation}
	T_0 \sim T_{thermal} + m_0 \cdot \left< \beta_T \right>^2
\end{equation}


Кроме того, из-за этой зависимости от скорости для более тяжелой системы столкновений, в которой предположительно более сильный коллективный поперечный поток, ожидается, что значение T0 будет больше. Таким образом, приведенные выше наблюдения качественно согласуются с гипотезой о поперечном гидродинамическом течении, возникающем при столкновениях тяжелых ионов. Количественно феноменологическая гидродинамическая модель, предложенная Schnedermann et. др. [13] можно применить к спектрам одиночных частиц для извлечения поперечной скорости и температуры при замораживании. В этой модели, называемой моделью «взрывной волны», эффекты коллективного расширения включаются в спектры поперечных масс следующим образом:

\begin{linenomath}
	\begin{equation}
		\frac{d \sigma}{m_T dm_T} \sim \int_0^R r dr \cdot m_T 
		I_0 \left(\frac{p_T sinh(\rho)}{T_{fo}}\right) K_1 \left(\frac{m_T cosh(\rho)}{T_{fo}} \right)
	\end{equation}
\end{linenomath}
где $I_0$ и $K_1$ представляют собой модифицированные функции Бесселя, где $\rho$ представляет собой поперечное усиление, которое зависит от радиального положения в соответствии с $\rho = tanh^{-1} \beta_r(r)$. Детали этого выражения описаны в Приложении A.2. Здесь $T_{fo}$ - температура замерзания, $R$ - максимальный радиус расширяющегося источника при замораживании. Профиль поперечной скорости $\beta_r(r)$ параметризуется как $\beta_r(r) = \beta_T(r/R)^n$ с поверхностной скоростью $\beta_T$. Мы можем варьировать форму профиля скорости с индексом $n$, например, $n = 0,5,1,2$. Средняя поперечная скорость определяется как
\begin{linenomath}
	\begin{equation}
		\left< \beta_T \right> = \frac{\int_0^R \beta_r(r)r dr}{\int_0^R r dr} = \left(\frac{2}{2+n} \right)\beta_T
	\end{equation}
\end{linenomath}
По результатам подгонки средняя поперечная скорость не зависит от профиля скорости [14]. На рис. 1.8 показаны результаты подгонки гидродинамической моделью течения при центральных столкновениях Pb + Pb и S+S в области средней скорости [11]. Сплошные линии — спектры источника с $T_{fo}$ = 140 МэВ и $\beta_T$ = 0,6 ($\left< \beta_T \right>$ = 0.4) для Pb+Pb и источника с $T_{fo}$ = 140 МэВ и $\beta_T$ = 0,41 ($\left< \beta_T \right>$ = 0,27) для S+S. Как показано на рис. 1.8, все спектры частиц от пионов до протонов очень хорошо воспроизводятся с двумя параметрами, $T_{fo}$ и $\beta_T$.

\section{Факторы ядерной модификации}
Для количественного сравнения выходов адронов в ядро-ядерных и протон-протонных соударениях были вычислены факторы ядерной модификации (\rab). Для столкновений ядер A и B значение \rab \ определяется соотношением
\begin{equation}
	\label{eq:rab}
	R_{AB}=\frac{1}{N_{coll}}\frac{d^2 N_{A+B}/dp_T dy}{d^2 N_{p+p}/dp_T dy} 
\end{equation}
где \Ncoll – количество парных нуклон-нуклонных соударений, 
$d^2 N_{A+B}/dp_T dy$ и $d^2 N_{p+p}/dp_T dy$ – инвариантные спектры адронов в столкновениях A+B и p+p соответственно.



