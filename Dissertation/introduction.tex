\chapter*{Введение}
\addcontentsline{toc}{chapter}{Введение}
Обзор, введение в тему, обозначение места данной работы в мировых исследованиях и т.п.

\textbf{Целью} данной работы является ...

Для~достижения поставленной цели необходимо было решить следующие задачи:
\begin{enumerate}
 \item Исследовать, разработать, вычислить и т.д. и т.п.
 \item Исследовать, разработать, вычислить и т.д. и т.п.
 \item Исследовать, разработать, вычислить и т.д. и т.п.
 \item Исследовать, разработать, вычислить и т.д. и т.п.
\end{enumerate}

%\newpage
\textbf{Основные положения, выносимые на~защиту:}
\begin{enumerate}
 \item Первое положение

 \item Второе положение

 \item Третье положение

 \item Четвертое положение
\end{enumerate}

\textbf{Научная новизна:}
\begin{enumerate}
  \item Впервые ...
  \item Впервые ...
  \item Было выполнено оригинальное исследование ...
\end{enumerate}

\textbf{Научная и практическая значимость} ...

\textbf{Степень достоверности} полученных результатов обеспечивается ... Результаты находятся в соответствии с результатами, полученными другими авторами.

\textbf{Апробация работы.}
Основные результаты работы докладывались~на:
перечисление основных конференций, симпозиумов и т.п.

\textbf{Личный вклад.} Автор принимал активное участие ...

\textbf{Публикации.} Основные результаты по теме диссертации изложены в ХХ печатных изданиях~\cite{}, %Martynova2007MAPATE, 
Х из которых изданы в журналах, рекомендованных ВАК~\cite{}, 
ХХ --- в тезисах докладов~\cite{}. %+ MartynovaKrupchatnikov2010 %Martynova2007MAPATE

{\bf Объем и структура работы.} Диссертация состоит из~введения, четырех глав, заключения и~двух приложений. Полный объем диссертации составляет ХХХ~страница с~ХХ~рисунками и~ХХ~таблицами. Список литературы содержит ХХХ~наименований.

\clearpage