{\actuality} Кварк-глюонная плазма (КГП) - это состояние вещества, которое существует при чрезвычайно высокой температуре ($>175$ МэВ) и плотности ($\sim 10$ Фм/$c$).  КГП состоит из асимптотически свободных сильно взаимодействующих кварков и глюонов, которые обычно находятся внутри атомных ядер или других адронов. Считается, что первые несколько микросекунд после Большого Взрыва вселенная находилась в состоянии КГП. Согласно квантовой хромодинамике, КГП может образовываться также и в столкновениях релятивистских тяжелых ионов. По мере расширения, КГП остывает и при достижении критической температуры происходит процесс адронизации - фазовый переход в нейтральную по цвету адронную материю.
\autocite{physica2020}

Одной из моделей адронизации является модель рекомбинации, согласно которой адроны образуются в результате объединения кварков, которые находятся рядом в фазовом пространстве. Поскольку связанные адронные состояния являются непертурбативными в рамках квантовой хромодинамики, описание процесса адронизации является чрезвычайно сложной задачей. В связи с этим особую важность приобретает анализ экспериментальных данных.

В 2002 году экспериментом PHENIX в столкновениях Au+Au при энергиях 130 ГэВ, а в последствии и при 200 ГэВ, было обнаружено аномально большое, по сравнению с протон-протонными столкновениями, отношение выходов (анти)протонов к выходам  \pipm-мезонов. В протон-протонных столкновениях в области поперечных импульсов \pt $\approx$3 ГэВ/с барионов рождается в 3 раза меньше, чем мезонов. Это связано с большими массами барионов и требованием ненулевого барионного числа для образования бариона. Однако экспериментом PHENIX было обнаружено, что в центральных Au+Au столкновениях барионы и мезоны рождаются примерно в равной пропорции. С увеличением центральности столкновений различие между результатами в протон-протонных взаимодействиях и Au+Au взаимодействиях уменьшается. Единственной моделью, способной объяснить данную аномалию оказалась модель рекомбинации – одна из моделей адронизации КГП.

Согласно Квантовой хромодинамике ожидалось, что в легких системах столкновений, таких как p+Al, \heau, d+Au, при энергии \sqsn=200 ГэВ условия, необходимые для образования КГП не достигаются. Однако в 2018 г экспериментом PHENIX были обнаружены эллиптические потоки заряженных частиц в p+Al, \heau, d+Au столкновениях, которые могут быть интерпретированы как признак образования КГП. 

Систематическое изучение рождения заряженных адронов в легких (p+Al, p/d/\heau) и тяжелых (Cu+Au, Au+Au и U+U) позволит изучить минимальные условия образования КГП
Таким образом, настоящая работа, посвященная исследованию особенностей рождения заряженных адрнов (\pipm, \Kpm, \prot, \aprot )в столкновениях p+Al, He+Au, Cu+Au   при \sqsn=200 ГэВ и в U+U столкновениях при \sqsn=193 ГэВ актуальна и является важной составляющей систематического изучения свойств КГП.

\begin{comment}
\ifsynopsis
Этот абзац появляется только в~автореферате.
Для формирования блоков, которые будут обрабатываться только в~автореферате,
заведена проверка условия \verb!\!\verb!ifsynopsis!.
Значение условия задаётся в~основном файле документа (\verb!synopsis.tex! для
автореферата).
\else
Этот абзац появляется только в~диссертации.
Через проверку условия \verb!\!\verb!ifsynopsis!, задаваемого в~основном файле
документа (\verb!dissertation.tex! для диссертации), можно сделать новую
команду, обеспечивающую появление цитаты в~диссертации, но~не~в~автореферате.
\fi
\end{comment}
% {\progress}
% Этот раздел должен быть отдельным структурным элементом по
% ГОСТ, но он, как правило, включается в описание актуальности
% темы. Нужен он отдельным структурынм элемементом или нет ---
% смотрите другие диссертации вашего совета, скорее всего не нужен.

{\aim} данной работы является изучение свойств  кварк-глюонной плазмы (температуры) путем измерения заряженных адронов в столкновениях \pal, \heau, \cuau\, при энергии \sqsn=200 ГэВ и в столкновениях \uu при энергии \sqsn=193 ГэВ.

Для~достижения поставленной цели необходимо было решить следующие {\tasks}:
\begin{enumerate}[beginpenalty=10000] % https://tex.stackexchange.com/a/476052/104425
 \item Измерить инвариантные спектры по поперечному импульсу для \pipm, \Kpm, \prot\,и \aprot \,в столкновениях  p+Al, $^{3}$He+Au, Cu+Au при энергии $\sqrt{s_{NN}}$=200 ГэВ и в столкновениях U+U при энергии $\sqrt{s_{NN}}$=193 ГэВ.
\item Измерить факторы ядерной модификации для \pipm, \Kpm, \prot и \aprot в столкновениях  p+Al, $^{3}$He+Au, Cu+Au при энергии $\sqrt{s_{NN}}$=200 ГэВ и в столкновениях U+U при энергии $\sqrt{s_{NN}}$=193 ГэВ.

\item Измерить отношения выходов \pim/\pip, \Km/\Kp, \prot/\aprot, \prot/\pip, \aprot/\pim, \Kp/\pip, \Km/\pim в столкновениях  p+Al, $^{3}$He+Au, Cu+Au при энергии $\sqrt{s_{NN}}$=200 ГэВ и в столкновениях U+U при энергии $\sqrt{s_{NN}}$=193 ГэВ.
\item Провести физическую интерпретацию результатов.
\end{enumerate}


{\novelty}
\begin{enumerate}[beginpenalty=10000] % https://tex.stackexchange.com/a/476052/104425
	\item Впервые измерены инвариантные спектры рождения по поперечному импульсу заряженных адронов (\pipm, \Kpm, \prot, \aprot) в столкновениях  p+Al, \heau, Cu+Au при энергии \sqsn=200 ГэВ и в столкновениях U+U при энергии \sqsn=193 ГэВ.
	\item Впервые получены факторы ядерной модификации для \pipm, \Kpm, \prot, \aprot\, в столкновениях  p+Al, \heau, Cu+Au при \sqsn=200 ГэВ и в столкновениях U+U при \sqsn=193 ГэВ
	\item 	Выпервые измерены отношения выходов \pim/\pip, \Km/\Kp, \prot/\aprot, \prot/\pip, \aprot/\pim, \Kp/\pip, \Km/\pim\,  в столкновениях  p+Al, \heau, Cu+Au при энергии \sqsn=200 ГэВ и в столкновениях U+U при энергии \sqsn=193 ГэВ.
\end{enumerate}

{\influence} \ldots

{\methods} \ldots

{\defpositions}
\begin{enumerate}[beginpenalty=10000] % https://tex.stackexchange.com/a/476052/104425
	\item 	Инвариантные спектры рождения по поперечному импульсу идентифицируемых заряженных адронов в столкновениях  p+Al,\heau, Cu+Au при энергии \sqsn =200 ГэВ и в столкновениях U+U при энергии \sqsn=193 ГэВ. \autocite{Gosele1999161,Lermontov}
	\item 	факторы ядерной модификации для \pipm, \Kpm, \prot, \aprot\, в столкновениях  p+Al, \heau, Cu+Au при \sqsn=200 ГэВ и в столкновениях U+U при \sqsn=193 ГэВ
		\begin{enumerate} 
			\item 	Значения факторов ядерной модификации, измеренные для одного типа частиц в различных системах столкновений при одинаковых значениях \Npart, совпадают в пределах погрешностей.
			\item В тяжелых системах столкновений  (Cu+Au, U+U) факторы ядерной модификации протонов повышены.
			\item В тяжелых системах столкновений  (Cu+Au, U+U) факторы ядерной модификации заряженных адронов проявляют зависимость от центральности. В периферических столкновениях значения \rab всех легких адронов 
			\item В столкновениях \heau  факторы ядерной модификации протонов повышены, в то время как факторы ядерной модификации мезонов близки к единице и совпадают в пределах погрешностей.
			\item 	В столкновениях p+Al значения \rab всех легких адронов (\pipm, \Kpm, \prot, \aprot, $\phi$, $K_s$, $K^{*0}$, $\eta$, $\omega$, $\pi^0$) близки к единице и совпадают в пределах погрешностей.
			\item 	Зависимость значений \rab от поперечного импульса (\pt), измеренная для для \Kpm, \pipm  в столкновениях p+Al имеет меньший угол наклона, чем в столкновениях \heau.
		\end{enumerate}
	
	\item Отношения выходов \pim/\pip, \Km/\Kp, \prot/\aprot, \prot/\pip, \aprot/\pim, \Kp/\Km в столкновениях  p+Al, $^{3}$He+Au, Cu+Au при энергии $\sqrt{s_{NN}}$=200 ГэВ и в столкновениях U+U при энергии $\sqrt{s_{NN}}$=193 ГэВ.
		\begin{enumerate}
			\item 	В тяжелых системах (Cu+Au, U+U) столкновений отношения \prot/\pip   и \aprot/\pim проявляют зависимость от центральности. 
			\item 	В легких системах столкновений (He+Au, p+Al) зависимость отношений \prot/\pip   и \aprot/\pim - от центральности не наблюдается
			\item 	Зависимость от центральности отношений \Kp/\pip и \Km/\pim не наблюдается во всех рассматриваемых системах столкновений.
		\end{enumerate}

\end{enumerate}


{\reliability} Измерения выходов заряженных адронов выполнялись согласно методике, принятой в коллаборации PHENIX. Результаты обсуждались на семинарах коллаборации PHENIX и международных конференциях «Ядро-2020» (Дубна, РФ), «Ядро-2021» (Санкт-Петербург, РФ) «PhysicA.SPb 2020» (Санкт-Петербург, РФ), «PhysicA.SPb 2021» (Санкт-Петербург, РФ), «PhysicA.SPb 2022» (Санкт-Петербург, РФ), «Lomonosov Conference» (Москва, РФ), «ICPPA-2020», (Москва, РФ). Результаты опубликованы в рецензируемых журналах баз ВАК и SCOPUS/WebOfScience.


{\probation}
Основные результаты работы докладывались~на:
Основные результаты работы докладывались~на:
международных конференциях «Ядро-2020» (Дубна, РФ), «Ядро-2021» (Санкт-Петербург, РФ) «PhysicA.SPb 2020» (Санкт-Петербург, РФ), «PhysicA.SPb 2021» (Санкт-Петербург, РФ), «PhysicA.SPb 2022» (Санкт-Петербург, РФ), «Lomonosov Conference» (Москва, РФ), «ICPPA-2020», (Москва, РФ). Результаты опубликованы в рецензируемых журналах баз ВАК и SCOPUS/WebOfScience.


{\contribution} Автор внес определяющий вклад в работу по отбору и анализу данных, получению физического результата и оценке систематических неопределенностей измерений.

\ifnumequal{\value{bibliosel}}{0}
{%%% Встроенная реализация с загрузкой файла через движок bibtex8. (При желании, внутри можно использовать обычные ссылки, наподобие ``).
    {\publications} Основные результаты по теме диссертации изложены
    в~XX~печатных изданиях,
    X из которых изданы в журналах, рекомендованных ВАК,
    X "--- в тезисах докладов.
}%
{%%% Реализация пакетом biblatex через движок biber
    \begin{refsection}[bl-author, bl-registered]
        % Это refsection=1.
        % Процитированные здесь работы:
        %  * подсчитываются, для автоматического составления фразы "Основные результаты ..."
        %  * попадают в авторскую библиографию, при usefootcite==0 и стиле `\insertbiblioauthor` или `\insertbiblioauthorgrouped`
        %  * нумеруются там в зависимости от порядка команд `\printbibliography` в этом разделе.
        %  * при использовании `\insertbiblioauthorgrouped`, порядок команд `\printbibliography` в нём должен быть тем же (см. biblio/biblatex.tex)
        %
        % Невидимый библиографический список для подсчёта количества публикаций:
        \printbibliography[heading=nobibheading, section=1, env=countauthor,          keyword=biblioauthorvak]%
        \printbibliography[heading=nobibheading, section=1, env=countauthorwos,          keyword=biblioauthorwos]%
        \printbibliography[heading=nobibheading, section=1, env=countauthorscopus,       keyword=biblioauthorscopus]%
        \printbibliography[heading=nobibheading, section=1, env=countauthorconf,         keyword=biblioauthorconf]%
        \printbibliography[heading=nobibheading, section=1, env=countauthorother,        keyword=biblioauthorother]%
        \printbibliography[heading=nobibheading, section=1, env=countregistered,         keyword=biblioregistered]%
        \printbibliography[heading=nobibheading, section=1, env=countauthorpatent,       keyword=biblioauthorpatent]%
        \printbibliography[heading=nobibheading, section=1, env=countauthorprogram,      keyword=biblioauthorprogram]%
        \printbibliography[heading=nobibheading, section=1, env=countauthor,             keyword=biblioauthor]%
        \printbibliography[heading=nobibheading, section=1, env=countauthorvakscopuswos, filter=vakscopuswos]%
        \printbibliography[heading=nobibheading, section=1, env=countauthorscopuswos,    filter=scopuswos]%
        %
        %\nocite{confbib1}
        %
        {\publications} Основные результаты по теме диссертации изложены в~\arabic{citeauthor}~печатных изданиях,
        \arabic{citeauthorvak} из которых изданы в журналах, рекомендованных ВАК\sloppy%
        \ifnum \value{citeauthorscopuswos}>0%
            , \arabic{citeauthorscopuswos} "--- в~периодических научных журналах, индексируемых Web of~Science и Scopus\sloppy%
        \fi%
        \ifnum \value{citeauthorconf}>0%
            , \arabic{citeauthorconf} "--- в~тезисах докладов.
        \else%
            .
        \fi%
        \ifnum \value{citeregistered}=1%
            \ifnum \value{citeauthorpatent}=1%
                Зарегистрирован \arabic{citeauthorpatent} патент.
            \fi%
            \ifnum \value{citeauthorprogram}=1%
                Зарегистрирована \arabic{citeauthorprogram} программа для ЭВМ.
            \fi%
        \fi%
        \ifnum \value{citeregistered}>1%
            Зарегистрированы\ %
            \ifnum \value{citeauthorpatent}>0%
            \formbytotal{citeauthorpatent}{патент}{}{а}{}\sloppy%
            \ifnum \value{citeauthorprogram}=0 . \else \ и~\fi%
            \fi%
            \ifnum \value{citeauthorprogram}>0%
            \formbytotal{citeauthorprogram}{программ}{а}{ы}{} для ЭВМ.
            \fi%
        \fi%
        % К публикациям, в которых излагаются основные научные результаты диссертации на соискание учёной
        % степени, в рецензируемых изданиях приравниваются патенты на изобретения, патенты (свидетельства) на
        % полезную модель, патенты на промышленный образец, патенты на селекционные достижения, свидетельства
        % на программу для электронных вычислительных машин, базу данных, топологию интегральных микросхем,
        % зарегистрированные в установленном порядке.(в ред. Постановления Правительства РФ от 21.04.2016 N 335)
    \end{refsection}%
    \begin{refsection}[bl-author, bl-registered]
        % Это refsection=2.
        % Процитированные здесь работы:
        %  * попадают в авторскую библиографию, при usefootcite==0 и стиле `\insertbiblioauthorimportant`.
        %  * ни на что не влияют в противном случае
        \nocite{physica2020}%vak
    \end{refsection}%
        %
        % Всё, что вне этих двух refsection, это refsection=0,
        %  * для диссертации - это нормальные ссылки, попадающие в обычную библиографию
        %  * для автореферата:
        %     * при usefootcite==0, ссылка корректно сработает только для источника из `external.bib`. Для своих работ --- напечатает "[0]" (и даже Warning не вылезет).
        %     * при usefootcite==1, ссылка сработает нормально. В авторской библиографии будут только процитированные в refsection=0 работы.
}

\begin{comment}
При использовании пакета \verb!biblatex! будут подсчитаны все работы, добавленные
в файл \verb!biblio/author.bib!. Для правильного подсчёта работ в~различных
системах цитирования требуется использовать поля:
\begin{itemize}
        \item \texttt{authorvak} если публикация индексирована ВАК,
        \item \texttt{authorscopus} если публикация индексирована Scopus,
        \item \texttt{authorwos} если публикация индексирована Web of Science,
        \item \texttt{authorconf} для докладов конференций,
        \item \texttt{authorpatent} для патентов,
        \item \texttt{authorprogram} для зарегистрированных программ для ЭВМ,
        \item \texttt{authorother} для других публикаций.
\end{itemize}
Для подсчёта используются счётчики:
\begin{itemize}
        \item \texttt{citeauthorvak} для работ, индексируемых ВАК,
        \item \texttt{citeauthorscopus} для работ, индексируемых Scopus,
        \item \texttt{citeauthorwos} для работ, индексируемых Web of Science,
        \item \texttt{citeauthorvakscopuswos} для работ, индексируемых одной из трёх баз,
        \item \texttt{citeauthorscopuswos} для работ, индексируемых Scopus или Web of~Science,
        \item \texttt{citeauthorconf} для докладов на конференциях,
        \item \texttt{citeauthorother} для остальных работ,
        \item \texttt{citeauthorpatent} для патентов,
        \item \texttt{citeauthorprogram} для зарегистрированных программ для ЭВМ,
        \item \texttt{citeauthor} для суммарного количества работ.
\end{itemize}
% Счётчик \texttt{citeexternal} используется для подсчёта процитированных публикаций;
% \texttt{citeregistered} "--- для подсчёта суммарного количества патентов и программ для ЭВМ.

Для добавления в список публикаций автора работ, которые не были процитированы в
автореферате, требуется их~перечислить с использованием команды \verb!\nocite! в
\verb!Synopsis/content.tex!.
\end{comment}