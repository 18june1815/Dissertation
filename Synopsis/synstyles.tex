%%% Изображения %%%
\graphicspath{{images/}{Synopsis/images/}}         % Пути к изображениям

%%% Макет страницы %%%
\geometry{a5paper, top=14mm, bottom=14mm, inner=18mm, outer=10mm, footskip=5mm, nomarginpar}%, showframe
\setlength{\topskip}{0pt}   %размер дополнительного верхнего поля

%%% Интервалы %%%
%% Реализация средствами класса (на основе setspace) ближе к типографской классике.
%% И правит сразу и в таблицах (если со звёздочкой)
%\DoubleSpacing*     % Двойной интервал
%\OnehalfSpacing*    % Полуторный интервал
\SingleSpacing      % Одинарный интервал
%\setSpacing{1.42}   % Полуторный интервал, подобный Ворду (возможно, стоит включать вместе с предыдущей строкой)

%%% Выравнивание и переносы %%%
%% http://tex.stackexchange.com/questions/241343/what-is-the-meaning-of-fussy-sloppy-emergencystretch-tolerance-hbadness
%% http://www.latex-community.org/forum/viewtopic.php?p=70342#p70342
\tolerance 1414
\hbadness 1414
\emergencystretch 1.5em % В случае проблем регулировать в первую очередь
\hfuzz 0.3pt
\vfuzz \hfuzz
%\raggedbottom
%\sloppy                 % Избавляемся от переполнений
\clubpenalty=10000      % Запрещаем разрыв страницы после первой строки абзаца
\widowpenalty=10000     % Запрещаем разрыв страницы после последней строки абзаца

%%% Колонтитулы %%%
\makeevenhead{plain}{}{}{}
\makeoddhead{plain}{}{}{}
\makeevenfoot{plain}{}{\thepage}{}
\makeoddfoot{plain}{}{\thepage}{}
\pagestyle{plain}

%%% Размеры заголовков %%%
\setsecheadstyle{\normalfont\large\bfseries}
\renewcommand*{\chaptitlefont}{\normalfont\large\bfseries}

%%% Подписи %%%
\setfloatadjustment{table}{%
    \setlength{\abovecaptionskip}{0pt}   % Отбивка над подписью
    \setlength{\belowcaptionskip}{0pt}   % Отбивка под подписью
}

%%% Отступы у плавающих блоков %%%
\setlength\textfloatsep{1ex}
