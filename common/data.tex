%%% Основные сведения %%%
\newcommand{\thesisAuthorLastName}{Ларионова}
\newcommand{\thesisAuthorOtherNames}{Дарья Максимовна}
\newcommand{\thesisAuthorInitials}{Д.\,М.}
\newcommand{\thesisAuthor}             % Диссертация, ФИО автора
{%
    \texorpdfstring{% \texorpdfstring takes two arguments and uses the first for (La)TeX and the second for pdf
        \thesisAuthorLastName~\thesisAuthorOtherNames% так будет отображаться на титульном листе или в тексте, где будет использоваться переменная
    }{%
        \thesisAuthorLastName, \thesisAuthorOtherNames% эта запись для свойств pdf-файла. В таком виде, если pdf будет обработан программами для сбора библиографических сведений, будет правильно представлена фамилия.
    }
}
\newcommand{\thesisAuthorShort}        % Диссертация, ФИО автора инициалами
{\thesisAuthorInitials~\thesisAuthorLastName}
%\newcommand{\thesisUdk}                % Диссертация, УДК
%{\fixme{xxx.xxx}}
\newcommand{\thesisTitle}              % Диссертация, название
{Рождение заряженных адронов в столкновениях $p$+Al, $^{3}$He+Au, Cu+Au при энергии $\sqrt{s_{NN}}$~=~200 ГэВ и U+U при $\sqrt{s_{NN}}$~=~193 ГэВ}
\newcommand{\thesisSpecialtyNumber}    % Диссертация, специальность, номер
{1.3.15}
\newcommand{\thesisSpecialtyTitle}     % Диссертация, специальность, название (название взято с сайта ВАК для примера)
{Физика атомных ядер и элементарных частиц, физика высоких энергий}
%% \newcommand{\thesisSpecialtyTwoNumber} % Диссертация, вторая специальность, номер
%% {\fixme{XX.XX.XX}}
%% \newcommand{\thesisSpecialtyTwoTitle}  % Диссертация, вторая специальность, название
%% {\fixme{Теория и~методика физического воспитания, спортивной тренировки,
%% оздоровительной и~адаптивной физической культуры}}
\newcommand{\thesisDegree}             % Диссертация, ученая степень
{кандидата физико-математических наук}
\newcommand{\thesisDegreeShort}        % Диссертация, ученая степень, краткая запись
{канд. физ.-мат. наук}
\newcommand{\thesisCity}               % Диссертация, город написания диссертации
{Санкт-Петербург}
\newcommand{\thesisYear}               % Диссертация, год написания диссертации
{\the\year}
\newcommand{\thesisOrganization}       % Диссертация, организация
{Федеральное государственное автономное образовательное учреждение высшего образования <<Санкт-Петербургский политехнический университет Петра Великого <<СПбПУ>>}
\newcommand{\thesisOrganizationShort}  % Диссертация, краткое название организации для доклада
{НазУчДисРаб}

\newcommand{\thesisInOrganization}     % Диссертация, организация в предложном падеже: Работа выполнена в ...
{федеральном государственном автономном образовательном учреждении высшего образования <<Санкт-Петербургский политехнический университет Петра Великого (СПбПУ)>>}

%% \newcommand{\supervisorDead}{}           % Рисовать рамку вокруг фамилии
\newcommand{\supervisorFio}              % Научный руководитель, ФИО
{Бердников Ярослав Александрович}
\newcommand{\supervisorRegalia}          % Научный руководитель, регалии
{доктор физико-математических наук, профессор}
\newcommand{\supervisorFioShort}         % Научный руководитель, ФИО
{Я.\,А.~Бердников}
\newcommand{\supervisorRegaliaShort}     % Научный руководитель, регалии
{д-р~физ.-мат. наук, профессор}

\newcommand{\opponentOneFio}           % Оппонент 1, ФИО
{Гуськов Алексей Вячеславович}
\newcommand{\opponentOneRegalia}       % Оппонент 1, регалии
{доктор физико-математических наук}
\newcommand{\opponentOneJobPlace}      % Оппонент 1, место работы
{Объединенный институт ядерных исследований}
\newcommand{\opponentOneJobPost}       % Оппонент 1, должность
{Заместитель директора Лаборатории ядерных проблем им. В.П. Джелепова ОИЯИ по научной работе}

\newcommand{\opponentTwoFio}           % Оппонент 2, ФИО
{Лохтин Игорь Петрович}
\newcommand{\opponentTwoRegalia}       % Оппонент 2, регалии
{доктор физико-математических наук, профессор РАН}
\newcommand{\opponentTwoJobPlace}      % Оппонент 2, место работы
{НИИЯФ МГУ}
\newcommand{\opponentTwoJobPost}       % Оппонент 2, должность
{ведущий научный сотрудник}

%% \newcommand{\opponentThreeFio}         % Оппонент 3, ФИО
%% {\fixme{Фамилия Имя Отчество}}
%% \newcommand{\opponentThreeRegalia}     % Оппонент 3, регалии
%% {\fixme{кандидат физико-математических наук}}
%% \newcommand{\opponentThreeJobPlace}    % Оппонент 3, место работы
%% {\fixme{Основное место работы c длинным длинным длинным длинным названием}}
%% \newcommand{\opponentThreeJobPost}     % Оппонент 3, должность
%% {\fixme{старший научный сотрудник}}

\newcommand{\leadingOrganizationTitle} % Ведущая организация, дополнительные строки. Удалить, чтобы не отображать в автореферате
{Федеральное государственное бюджетное образовательное учреждение высшего образова­ния «Санкт-Петербургский государственный университет»}

\newcommand{\defenseDate}              % Защита, дата
{30 октября 2024~г.~в~16 час. 00 мин.}
\newcommand{\defenseCouncilNumber}     % Защита, номер диссертационного совета
{У\,1.3.3.09}
\newcommand{\defenseCouncilTitle}      % Защита, учреждение диссертационного совета
{Название учреждения}
\newcommand{\defenseCouncilAddress}    % Защита, адрес учреждение диссертационного совета
{ул. Политехническая, д. 29, Санкт-Петербург, 195251}
\newcommand{\defenseCouncilPhone}      % Телефон для справок
{+7~(921)~639-76-77}

\newcommand{\defenseSecretaryFio}      % Секретарь диссертационного совета, ФИО
{Орленко Елена Владимировна}
\newcommand{\defenseSecretaryRegalia}  % Секретарь диссертационного совета, регалии
{д-р~физ.-мат. наук, доцент}            % Для сокращений есть ГОСТы, например: ГОСТ Р 7.0.12-2011 + http://base.garant.ru/179724/#block_30000

\newcommand{\synopsisLibrary}          % Автореферат, название библиотеки
{ФГАОУ ВО <<СПбПУ>> и на сайте https://www.spbstu.ru/}
\newcommand{\synopsisDate}             % Автореферат, дата рассылки
{<<~~~~~>> сентября 2024~года}

% To avoid conflict with beamer class use \providecommand
\providecommand{\keywords}%            % Ключевые слова для метаданных PDF диссертации и автореферата
{}
