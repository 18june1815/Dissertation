%%% Макет страницы %%%
% Выставляем значения полей (ГОСТ 7.0.11-2011, 5.3.7)
\geometry{a4paper,top=2cm,bottom=2cm,left=2.5cm,right=1cm}

%%% Кодировки и шрифты %%%
\ifxetex
  \setmainlanguage[babelshorthands=true]{russian}   % Язык по-умолчанию русский с поддержкой приятных команд пакета babel
  \setotherlanguage{english}                        % Дополнительный язык = английский (в американской вариации по-умолчанию)
  \defaultfontfeatures{Ligatures=TeX,Mapping=tex-text}
  \setmainfont{Times New Roman}
  \newfontfamily\cyrillicfont{Times New Roman}
  \setsansfont{Arial}
  \newfontfamily\cyrillicfontsf{Arial}
  \setmonofont{Courier New}
  \newfontfamily\cyrillicfonttt{Courier New}
\else
  \IfFileExists{pscyr.sty}{\renewcommand{\rmdefault}{ftm}}{}
\fi

%%% Выравнивание и переносы %%%
\sloppy                             % Избавляемся от переполнений
\clubpenalty=10000                  % Запрещаем разрыв страницы после первой строки абзаца
\widowpenalty=10000                 % Запрещаем разрыв страницы после последней строки абзаца

%%% Библиография %%%
\makeatletter
\bibliographystyle{../BibTeX-Styles/utf8gost71u}     % Оформляем библиографию по ГОСТ 7.1 (ГОСТ Р 7.0.11-2011, 5.6.7)
\renewcommand{\@biblabel}[1]{#1.}   % Заменяем библиографию с квадратных скобок на точку
\makeatother
%% Управление отступами между записями
%% требует etoolbox
%% http://tex.stackexchange.com/a/105642
%\patchcmd\thebibliography
% {\labelsep}
% {\labelsep\itemsep=5pt\parsep=0pt\relax}
% {}
% {\typeout{Couldn't patch the command}}

%%% Цитирование %%%
\renewcommand\citepunct{;\penalty\citepunctpenalty%
    \hskip.13emplus.1emminus.1em\relax}                % Разделение ; при перечислении ссылок (ГОСТ Р 7.0.5-2008)

%%% Изображения %%%
\graphicspath{{images/}}            % Пути к изображениям

%%% Рисунки %%%
\DeclareCaptionLabelSeparator*{emdash}{~--- }             % (ГОСТ 2.105, 4.3.1)
\captionsetup[figure]{labelsep=emdash,font=onehalfspacing,position=bottom}

%%% Таблицы %%%
\DeclareCaptionFormat{tablecaption}{\raggedleft #1#2\\%   % Идентификатор таблицы справа, на отдельной строке
    \centering{#3}}                                           % Наименование таблицы строкой ниже и центрировано, без переносов
\DeclareCaptionFormat{tablenocaption}{\raggedleft #1#2%   % Идентификатор таблицы справа, на отдельной строке
}                                                         % Наименование таблицы отсутствует
\captionsetup[table]{format=tablecaption,labelsep=space,singlelinecheck=off,font=onehalfspacing,position=top}  % пробельный разделитьель идентификатора с номером от наименования, многострочные наименования и прочее
\DeclareCaptionLabelFormat{continued}{Продолжение таблицы~#2}

%%% Подписи %%%
\renewcommand{\thesubfigure}{\asbuk{subfigure}}           % Буквенные номера подрисунков
\captionsetup[subfloat]{font={normalsize},                % Шрифт подписи названий подрисунков (не отличается от основного)
    labelformat=brace,                                    % Формат обозначения подрисунка
    justification=centering                               % Выключка подписей (форматирование), один из вариантов
}
%\DeclareCaptionFont{font12pt}{\fontsize{12pt}{13pt}\selectfont} % объявляем шрифт 12pt для использования в подписях, тут же надо интерлиньяж объявлять, если не наследуется
%\captionsetup[subfloat]{font={font12pt}}                  % Шрифт подписи названий подрисунков (всегда 12pt)

%%% Цвета гиперссылок %%%
\definecolor{linkcolor}{rgb}{0.9,0,0}
\definecolor{citecolor}{rgb}{0,0.6,0}
\definecolor{urlcolor}{rgb}{0,0,1}
\hypersetup{
    colorlinks, linkcolor={linkcolor},
    citecolor={citecolor}, urlcolor={urlcolor}
}

%%% Оглавление %%%
\renewcommand{\cftchapdotsep}{\cftdotsep}

\renewcommand{\cfttoctitlefont}{\hfill\fontsize{14pt}{16pt}\selectfont\bfseries}% вместе со следующей строкой
\renewcommand{\cftaftertoctitle}{\hfill}% устанавливает заголовок по центру.
\setlength{\cftbeforetoctitleskip}{-1.4\curtextsize}% Поскольку этот заголовок всегда является первым на странице, то перед ним отделять пустым тройным интервалом не следует. Независимо от основного шрифта, в этом случае зануление (почти) происходит при -1.4\curtextsize.
\setlength{\cftaftertoctitleskip}{3\curtextsize}% Если считаем Оглавление заголовком, то выставляем после него тройной интервал через наше определённое значение

%% Ниже закомментирован вариант, несоответствующий ГОСТ 7.0.11, в котором заголовки крупнее положенного размера в 12-14 пт
%\ifnum\curtextsize>\bigtextsize                     % Проверяем условие использования базового шрифта 14 pt
%\renewcommand{\cfttoctitlefont}{\LARGE\bfseries}    % Исправляем размер заголовка оглавления
%\fi

%% Переносить слова в заголовке не допускается (ГОСТ Р 7.0.11-2011, 5.3.5). Заголовки в оглавлении должны точно повторять заголовки в тексте (ГОСТ Р 7.0.11-2011, 5.2.3). Прямого указания на запрет переносов в оглавлении нет, но по той же логике невнесения искажений в смысл, лучше в оглавлении не переносить:
\cftsetrmarg{2.55em plus1fil} %To have the (sectional) titles in the ToC, etc., typeset ragged right with no hyphenation
\renewcommand{\cftchappagefont}{\normalfont}        % нежирные номера страниц у глав в оглавлении
\renewcommand{\cftchapleader}{\cftdotfill{\cftchapdotsep}} % нежирные точки до номеров страниц у глав в оглавлении
%\renewcommand{\cftchapfont}{}                      % нежирные названия глав в оглавлении

%\addtocontents{toc}{~\hfill{Стр.}\par}			    % добавить Стр. над номерами страниц


%%% Оформление названий глав %%%
%% настройки заголовка списка рисунков
\renewcommand{\cftloftitlefont}{\hfill\fontsize{14pt}{16pt}\selectfont\bfseries}% вместе со следующей строкой
\renewcommand{\cftafterloftitle}{\hfill}% устанавливает заголовок по центру.
\setlength{\cftbeforeloftitleskip}{-1.5\curtextsize}% Поскольку этот заголовок всегда является первым на странице, то перед ним отделять пустым тройным интервалом не следует. Независимо от основного шрифта, в этом случае зануление (почти) происходит при -1.5\curtextsize.
\setlength{\cftafterloftitleskip}{3\curtextsize}% выставляем после него тройной интервал через наше определённое значение

%% настройки заголовка списка таблиц
\renewcommand{\cftlottitlefont}{\hfill\fontsize{14pt}{16pt}\selectfont\bfseries}% вместе со следующей строкой
\renewcommand{\cftafterlottitle}{\hfill}% устанавливает заголовок по центру.
\setlength{\cftbeforelottitleskip}{-1.5\curtextsize}% Поскольку этот заголовок всегда является первым на странице, то перед ним отделять пустым тройным интервалом не следует. Независимо от основного шрифта, в этом случае зануление (почти) происходит при -1.5\curtextsize.
\setlength{\cftafterlottitleskip}{3\curtextsize}% выставляем после него тройной интервал через наше определённое значение

%% Ниже закомментирован вариант, несоответствующий ГОСТ 7.0.11, в котором заголовки крупнее положенного размера в 12-14 пт
\ifnum\curtextsize>\bigtextsize                     % Проверяем условие использования базового шрифта 14 pt
%\renewcommand{\cftloftitlefont}{\LARGE\bfseries}    % Исправляем размер заголовка списка рисунков
%\renewcommand{\cftlottitlefont}{\LARGE\bfseries}    % Исправляем размер заголовка списка таблиц
%\titleformat*{\chapter}{\LARGE\bfseries}            % Исправляем размер заголовка названий глав
\setlength{\headheight}{17pt}                       % Исправляем высоту заголовка
\else
\setlength{\headheight}{15pt}                       % Исправляем высоту заголовка
\fi

%%% Шаблон %%%
\newcommand{\todo}[1]{\textcolor{red}{#1}}
\setlength{\parindent}{2.5em}                       % Абзацный отступ. Должен быть одинаковым по всему тексту и равен пяти знакам (ГОСТ Р 7.0.11-2011, 5.3.7).

%%% Списки %%%
% Используем дефис для ненумерованных списков (ГОСТ 2.105-95, 4.1.7)
\renewcommand{\labelitemi}{\normalfont\bfseries{--}}
\setlist{nosep,%                                    % Единый стиль для всех списков (пакет enumitem), без дополнительных интервалов.
    labelindent=\parindent,leftmargin=*%            % Каждый пункт, подпункт и перечисление записывают с абзацного отступа (ГОСТ 2.105-95, 4.1.8)
}

%%% Колонтитулы %%%
% Порядковый номер страницы печатают на середине верхнего поля страницы (ГОСТ Р 7.0.11-2011, 5.3.8)
\makeatletter
\let\ps@plain\ps@fancy              % Подчиняем первые страницы каждой главы общим правилам
\makeatother
\pagestyle{fancy}                   % Меняем стиль оформления страниц
\fancyhf{}                          % Очищаем текущие значения
\fancyhead[C]{\thepage}             % Печатаем номер страницы на середине верхнего поля
\renewcommand{\headrulewidth}{0pt}  % Убираем разделительную линию

%%% Оформление заголовков глав, разделов, подразделов %%%
%% Работа должна быть выполнена ... размером шрифта 12-14 пунктов (ГОСТ Р 7.0.11-2011, 5.3.8). То есть не должно быть надписей шрифтом более 14. Так и поставим.
%% Эти установки будут давать одинаковый результат независимо от выбора базовым шрифтом 12 пт или 14 пт
\titleformat{\chapter}[block]% default display;  hang = with a hanging label. (Like the standard \section.); block = typesets the whole title in a block (a paragraph) without additional formatting. Useful in centered titles
        {\filcenter\fontsize{14pt}{16pt}\selectfont\bfseries}%\filright - по левому краю; \filcenter - по центру. В обоих случаях, вроде бы без переноса, как и надо (ГОСТ Р 7.0.11-2011, 5.3.5)
        %\fontsize{<size>}{<skip>} % второе число ставим 1.2*первое, чтобы адекватно отрабатывали команды по расчету полуторного интервала (домножая разные комбинации коэффициентов на этот)
        {\thechapter}% Заголовки в оглавлении должны точно повторять заголовки в тексте (ГОСТ Р 7.0.11-2011, 5.2.3). Раз не фигурирует слово Глава в оглавлении, то и здесь не надо
        {1em}%
        {}%

\titleformat{\section}[block]% default hang;  hang = with a hanging label. (Like the standard \section.); block = typesets the whole title in a block (a paragraph) without additional formatting. Useful in centered titles
        {\filcenter\fontsize{14pt}{16pt}\selectfont\bfseries}%\filright - по левому краю; \filcenter - по центру. В обоих случаях, вроде бы без переноса, как и надо (ГОСТ Р 7.0.11-2011, 5.3.5)
        %\fontsize{<size>}{<skip>} % второе число ставим 1.2*первое, чтобы адекватно отрабатывали команды по расчету полуторного интервала (домножая разные комбинации коэффициентов на этот)
        {\thesection}% Заголовки в оглавлении должны точно повторять заголовки в тексте (ГОСТ Р 7.0.11-2011, 5.2.3).
        {1em}%
        {}%

\titleformat{\subsection}[block]% default hang;  hang = with a hanging label. (Like the standard \section.); block = typesets the whole title in a block (a paragraph) without additional formatting. Useful in centered titles
        {\filcenter\fontsize{14pt}{16pt}\selectfont\bfseries}%\filright - по левому краю; \filcenter - по центру. В обоих случаях, вроде бы без переноса, как и надо (ГОСТ Р 7.0.11-2011, 5.3.5)
        %\fontsize{<size>}{<skip>} % второе число ставим 1.2*первое, чтобы адекватно отрабатывали команды по расчету полуторного интервала (домножая разные комбинации коэффициентов на этот)
        {\thesubsection}% Заголовки в оглавлении должны точно повторять заголовки в тексте (ГОСТ Р 7.0.11-2011, 5.2.3).
        {1em}%
        {}%

%% Интервалы между заголовками
% На эти величины titlespacing множит через *
\beforetitleunit=\curtextsize% привязались к нашему полуторному интервалу, поскольку он к компилятору не привязан
\aftertitleunit=\curtextsize% привязались к нашему полуторному интервалу, поскольку он к компилятору не привязан

\titlespacing{\chapter}{0pt}{-1.7em}{*3}            % Заголовки отделяют от текста сверху и снизу тремя интервалами (ГОСТ Р 7.0.11-2011, 5.3.5). Поскольку название главы всегда является первым на странице, то перед ним отделять пустым тройным интервалом не следует. Независимо от основного шрифта, в этом случае зануление происходит при -1.7em.
\titlespacing{\section}{0pt}{*3}{*3}                % Заголовки отделяют от текста сверху и снизу тремя интервалами (ГОСТ Р 7.0.11-2011, 5.3.5)
\titlespacing{\subsection}{0pt}{*3}{*3}             % Заголовки отделяют от текста сверху и снизу тремя интервалами (ГОСТ Р 7.0.11-2011, 5.3.5)
\titlespacing{\subsubsection}{0pt}{*3}{*3}          % Заголовки отделяют от текста сверху и снизу тремя интервалами (ГОСТ Р 7.0.11-2011, 5.3.5)
%\parindent вместо 0pt в настройке отступов для подразделов (четыре строки выше), заставит их выравниваться слева по абзацному отступу (полезно, если выше вы поставите выравнивание по левому краю)

%%% Счётчики %%%
%% Счётчик использованных ссылок на литературу, обрабатывающий с учётом неоднократных ссылок
%% Требуется дважды компилировать, поскольку ему нужно считать актуальный внешний файл со списком литературы
\newtotcounter{citenum}
\def\oldcite{}
\let\oldcite=\bibcite
\def\bibcite{\stepcounter{citenum}\oldcite}

%%http://www.linux.org.ru/forum/general/6993203#comment-6994589 (используется totcount)
\makeatletter
\def\formbytotal#1#2#3#4#5{%
    \newcount\@c
    \@c\totvalue{#1}\relax
    \newcount\@last
    \newcount\@pnul
    \@last\@c\relax
    \divide\@last 10
    \@pnul\@last\relax
    \divide\@pnul 10
    \multiply\@pnul-10
    \advance\@pnul\@last
    \multiply\@last-10
    \advance\@last\@c
    \total{#1}~#2%
    \ifnum\@pnul=1#5\else%
    \ifcase\@last#5\or#3\or#4\or#4\or#4\else#5\fi
    \fi
}
\makeatother
