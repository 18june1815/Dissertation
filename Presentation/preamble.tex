\begin{frame}[noframenumbering,plain]
    \setcounter{framenumber}{1}
    \maketitle
\end{frame}

\begin{frame}
    \frametitle{Положения, выносимые на защиту}
    \begin{itemize}
       	\item Впервые измеренные инвариантные спектры по поперечному импульсу, факторы ядерной модификации идентифицируемых заряженных адронов ($\pi^{\pm}$, $K^{\pm}$, $p$, $\bar{p}$), а также величины отношений выходов $\pi^{-}/\pi^{+}$, $K^{-}/\pi^{+}$, $\bar{p}/p$, $K^{+}/\pi^{+}$, $K^{-}/\pi^{-}$, $p/\pi^{+}$, $\bar{p}/\pi^{-}$ в столкновениях  $p$+Al, $^{3}$He+Au, Cu+Au при энергии $\sqrt{s_{NN}}$ = 200 ГэВ и в столкновениях U+U при энергии $\sqrt{s_{NN}}$ = 193 ГэВ. 
       
       \item Значения температуры химического вымораживания $T_0$ и скоростей коллективного потока частиц $\left< u_{t} \right>$, измеренные как функция от количества нуклонов-участников $\left< N_{part} \right>$ в $p$+Al, $^{3}$He+Au, Cu+Au и U+U столкновениях.
       
       \item Особенности рождения $\pi^{\pm}$, $K^{\pm}$, $p$, $\bar{p}$ в $^{3}$He+Au, Cu+Au, U+U  столкновениях не зависят от геометрии области перекрытия сталкивающихся ядер, а определяются количеством нуклонов-участников.
        \end{itemize}
 \end{frame}  
 
 \begin{frame}
 	\frametitle{Положения, выносимые на защиту}
 	\begin{itemize}
       \item В центральных столкновениях $^{3}$He+Au, Cu+Au, U+U наблюдается эффект увеличенного выхода протонов и антипротонов, что может быть объяснено доминированием вклада процессов рекомбинации в образовние иднентифицируемых заряженных адронов в диапазоне малых и промежуточных поперечных импульсов ($p_{T}<4$ ГэВ/$c$). 
       
       \item В $p$+Al столкновениях, а также в периферических столкновениях $^{3}$He+Au, Cu+Au, U+U эффект увеличенного выхода протонов и антипротонов не наблюдается, что может быть объяснено доминированием вклада процессов фрагментации в образовние иднентифицируемых заряженных адронов в диапазоне промежуточных поперечных импульсов (2 ГэВ/$c$ $<p_{T}<4$ ГэВ/$c$).
    \end{itemize}
\end{frame}
\note{
    Проговариваются вслух положения, выносимые на защиту
}

\begin{frame}
    \frametitle{Содержание}
    \tableofcontents
\end{frame}
\note{
    Работа состоит из четырёх глав.

    \medskip
    В первой главе \dots

    Во второй главе \dots

    Третья глава посвящена \dots

    В четвёртой главе \dots
}
